\documentclass{article}
\usepackage[utf8]{inputenc}
\usepackage{amsfonts}
\usepackage{amsmath}
\usepackage{amsthm}
\usepackage{amssymb}
\usepackage{mathtools}
\usepackage{tikz}
\usepackage{quiver}
\usepackage{tikz-cd}
\usepackage{bbm}
\usepackage{graphicx}
\usepackage{enumitem}

\newcommand{\R}{\mathbb R}
\newcommand{\N}{\mathbb N}
\newcommand{\Q}{\mathbb Q}
\newcommand{\Z}{\mathbb Z}
\newcommand{\C}{\mathbb C}
\newcommand{\HH}{\mathcal H}
\newcommand{\posRcl}{{\mathbb R}_{\geq 0}}
\newcommand{\one}{\mathbbm{1}}
\newcommand{\g}{\mathfrak{g}}

\newcommand{\eps}{\varepsilon}
\newcommand{\nl}{\newline\newline\noindent}
\newcommand{\cpt}{[0,1]}
\newcommand{\bI}{\mathbf{I}}
\newcommand{\xv}{\vec{x}}
\newcommand{\yv}{\vec{y}}
\newcommand{\al}{\alpha}
\newcommand{\be}{\beta}
\newcommand{\ga}{\gamma}
\newcommand{\de}{\delta}
\newcommand{\topo}{{T}}
\newcommand{\vhi}{\varphi}
\newcommand{\rank}{\text{rank}}
\newcommand{\sgn}{\text{sgn}}
\newcommand{\w}{\omega}
\newcommand{\pd}[1]{\frac{\partial}{\partial #1}}
\newcommand{\pdof}[2]{\frac{\partial #1}{\partial #2}}
\newcommand{\inv}[1]{#1^{-1}}
\newcommand{\bigslant}[2]{\left.\raisebox{.1em}{$#1$}\middle/\raisebox{-.15em}{$#2$}\right.}
\newcommand{\Int}{\text{Int}}
\newcommand{\im}{\text{im}\,}
\newcommand{\Hopf}{\text{Hopf}\,}
\newcommand{\bra}{\langle}
\newcommand{\ket}{\rangle}
\DeclareMathOperator{\supp}{supp}
\DeclareMathOperator{\spn}{span}
\DeclareMathOperator{\Hom}{Hom}
\DeclareMathOperator{\tr}{tr}
\DeclarePairedDelimiter{\ang}{\langle}{\rangle}
\title{MATH 689 - Physics for Mathematicians}
\author{Lectures by Igor Zelenko, transcribed by Seth Hoisington}
\date{September 14, 2023}

\newtheorem{thm}{Theorem}
\newtheorem{ex}{Example}
\newtheorem{defn}{Definition}
\newtheorem{lem}{Lemma}
\newtheorem{cor}{Corollary}
\newtheorem{rk}{Remark}

\begin{document}

\maketitle
\section*{Special Relativity}
\subsection*{Fundamental principles of special relativity}
\begin{enumerate}[label=\Alph*]
    \item \textbf{Relativity principle in inertial frames:} All laws of nature and equations describing them are the same in all inertial frames. In other words, all inertial frames are identical.
    \item \textbf{Independence of speed of light w.r.t the motion of a source in an inertial frame}. The speed of light in a vacuum is independent of the motion of the source and is the same in all directions
\end{enumerate}
Let $K$ and $K'$ be two inertial frames such that $K$ moves along the $x$-axis with speed $v$. We want to find the transformation between 2 space-time coordinate systems $(x,t)$ and $(x',t')$ with respect to $K$ and $K'$, and it turns out that these two principles alone are sufficient. We consider
\begin{equation}
    x' = x'(x,t)\quad\text{and}\quad t' = t'(x,t)
\end{equation}
Principles A and B imply that (1) must satisfy the following 4 properties:
\begin{itemize}
    \item It sends straight lines (trajectories of free particles) 
    \item 
    \item The transformation (1) corresponding to velocity $-v$ is the inverse of the transformation (1) corresponding to the velocity $v$ (as $K'$ moves with velocity $-v$ with respect to $K$, using the relativity principle again)
    \item 
\end{itemize}
From \ref{eqn1} it follows that it is linear with respect to $(x,t)$. So
\[\begin{pmatrix}
    x'\\
    t'
\end{pmatrix}=A(v)\begin{pmatrix}
    x\\
    t
\end{pmatrix}\]
From linearity and Property 2, it follows that 
\begin{equation}
    (ct')^2-\|x'\|^2 = \lambda(v)((ct)^2 - \|x\|^2)
\end{equation}
Besides $\lambda(0) = 1$ and 
\end{document}