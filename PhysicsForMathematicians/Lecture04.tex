\documentclass{article}
\usepackage[utf8]{inputenc}
\usepackage{amsfonts}
\usepackage{amsmath}
\usepackage{amsthm}
\usepackage{amssymb}
\usepackage{mathtools}
\usepackage{tikz}
\usepackage{quiver}
\usepackage{tikz-cd}
\usepackage{bbm}
\usepackage{graphicx}

\newcommand{\R}{\mathbb R}
\newcommand{\N}{\mathbb N}
\newcommand{\Q}{\mathbb Q}
\newcommand{\Z}{\mathbb Z}
\newcommand{\C}{\mathbb C}
\newcommand{\HH}{\mathcal H}
\newcommand{\posRcl}{{\mathbb R}_{\geq 0}}
\newcommand{\one}{\mathbbm{1}}
\newcommand{\g}{\mathfrak{g}}

\newcommand{\eps}{\varepsilon}
\newcommand{\nl}{\newline\newline\noindent}
\newcommand{\cpt}{[0,1]}
\newcommand{\bI}{\mathbf{I}}
\newcommand{\xv}{\vec{x}}
\newcommand{\yv}{\vec{y}}
\newcommand{\al}{\alpha}
\newcommand{\be}{\beta}
\newcommand{\ga}{\gamma}
\newcommand{\de}{\delta}
\newcommand{\topo}{{T}}
\newcommand{\vhi}{\varphi}
\newcommand{\rank}{\text{rank}}
\newcommand{\sgn}{\text{sgn}}
\newcommand{\w}{\omega}
\newcommand{\pd}[1]{\frac{\partial}{\partial #1}}
\newcommand{\pdof}[2]{\frac{\partial #1}{\partial #2}}
\newcommand{\inv}[1]{#1^{-1}}
\newcommand{\bigslant}[2]{\left.\raisebox{.1em}{$#1$}\middle/\raisebox{-.15em}{$#2$}\right.}
\newcommand{\Int}{\text{Int}}
\newcommand{\im}{\text{im}\,}
\newcommand{\Hopf}{\text{Hopf}\,}
\newcommand{\bra}{\langle}
\newcommand{\ket}{\rangle}
\DeclareMathOperator{\supp}{supp}
\DeclareMathOperator{\spn}{span}
\DeclareMathOperator{\Hom}{Hom}
\DeclareMathOperator{\tr}{tr}
\DeclarePairedDelimiter{\ang}{\langle}{\rangle}
\title{MATH 689 - Physics for Mathematicians}
\author{Lectures by Igor Zelenko, transcribed by Seth Hoisington}
\date{Ausust 31, 2023}

\newtheorem{thm}{Theorem}
\newtheorem{ex}{Example}
\newtheorem{defn}{Definition}
\newtheorem{lm}{Lemma}
\newtheorem{cor}{Corollary}
\newtheorem{rk}{Remark}
\newtheorem{prop}{Proposition}

\begin{document}

\maketitle
\section{Continued from last class:}
\begin{lm}
    For any Hamiltonian $H:T^*M\to \R$, there exists a unique vector field, denoted by $\vec{H}$ such that 
    \begin{equation}
        \iota_{\vec H} \sigma = -dH.
    \end{equation}
    Moreover, in the canonical coordinates $(p_1,\dots, p_n,q^1,\dots,q^n)\in T^*M$,
    \begin{equation}\label{eq:2}
        \vec H = \sum_{i} \pdof{H}{p_i}\pd{q^i} - \pdof{H}{q^i}\pd{p_i}.
    \end{equation}
    Thus, a curve $\lambda(t)$ in $T^*M$ satisfies $\lambda(t) = \vec H(\lambda(t))$ if and only if 
    \[\left\{\begin{aligned}
        \dot q^i(t) &= \pdof{H}{p_i}(p(t), q(t))\\
        \dot p^i(t) &= -\pdof{H}{q^i}(p(t), q(t))
    \end{aligned}\right.\]
\end{lm}
\begin{proof}
    (Continued from previous lecture) We shall now prove (\ref{eq:2}). Assume $\vec H = \sum_{i} v^i\pd{q^i} + w_i\pd{p_i}$ and let $z = \sum_{i} x_i\pd{q^i} + y_i\pd{p_i}$. Then
    \begin{align*}
        \sigma(\vec{H},z) &= \sum_i (dp_i\wedge dq^i)(\vec{H},z)\\
        &=\sum_i w_ix^i - v^iy_i.\\
        &=\sum_i w_idq^i(z) - v^idp_i(z).\\
    \end{align*}
    Therefore, $\iota_{\vec H} = \sum_i w_idq^i - v^idp_i$. Since
    \[-dH = \sum_i -\pdof{H}{q^i}dq^i - \pdof{H}{p_i}dp_i,\]
    we have that $w_i = -\pdof{H}{q^i}$ and $v^i = \pdof{H}{p_i}$.
\end{proof}
\begin{defn}
    $\vec H$ is called the \textbf{Hamiltonian vector field} associated to $H$.
\end{defn}
\section{Poisson Brackets}
Let $H,G\in C^\infty(T^*M)$. Then the Poisson Bracket $\{H,G\}$ is given by:
\[\{H,G\}:= \sigma(\vec{H},\vec G) = dG(\vec H) = \vec H(G)\]
\subsection{Properties of the Poisson Bracket:}
\begin{itemize}
    \item \textbf{Skew-symmetry:} $\{H,G\} = -\{G,H\}$. Follows from skew-symmetry of $\sigma$.
    \item \textbf{Jacobi Identity} Exercise 3 of Homework 1. Specifically, Since $\overrightarrow{\{H,G\}} = \left[\vec H, \vec G\right]$, then the Jacobi Identity for the Poisson bracket follows from the Lie bracket (??).
    \item Thus, the Poisson Bracket is a Lie bracket on $C^\infty(T^*M)$, making the space a (very big) Lie algebra!
\end{itemize}
\subsection{Coordinate Expression for Poisson Brackets in canonical coordinates}
We see that $\{H,G\} = \vec H(G) = \sum_i\pdof{H}{p_i}\pdof{G}{q^i} - \pdof{H}{q^i}\pdof{G}{p_i}$, we have that $\{p_i,p_j\} = \{q^i,q^j\} = 0$ and $\{p_i,q^j\} = \delta_{ij}$.
\section{First integrals in the Hamiltonian setting}
Now, we consider the following Hamiltonian system:
\[\left\{\begin{aligned}
    \dot q^i &= \{H,q^i\} = \pdof{H}{p_i} \\
    \dot p_i &= \{H,p_i\} = -\pdof{H}{q^i}
\end{aligned}\right.\]
Let us consider the first integrals of motion of this system.
\begin{defn}
    Akin to the Lagrangian case, a \textbf{trajectory} $\lambda(t)$ of the Hamiltonian system defined by $H$ is a curve in $T^*M$ which is a solution of the system $\dot\lambda = \vec H(\lambda(t))$.
\end{defn}
\begin{defn}
    A function $G:T^*M\to \R$ is called a first integral of the Hamiltonian system $\dot\lambda = \vec{H} \lambda$ if, for any trajectory $\lambda(t)$ of the system, $G(\lambda(t))$ is constant.
\end{defn}    
\begin{prop}
    $G$ is an integral of the Hamiltonian system defined by $H$ if and only if $\{G,H\} = 0$. 
\end{prop}
\begin{proof}
    First, we see that, in coordinates,
    \begin{align*}
        \frac{d}{dt}G(\lambda(t)) &= DG_{\lambda(t)}(\dot\lambda(t))\\
        &= DG_{\lambda(t)}\left(\vec H_{\lambda(t)}\right)\\
        &=\sum_i \pdof{G}{p_i}\left(\vec H_{\lambda(t)}\right)_{p_i} + \pdof{G}{q^i}\left(\vec H_{\lambda(t)}\right)_{q^i}\\
        &=\sum_i \pdof{G}{p_i}\left(-\pdof{H}{q^i}\right)_{p_i} + \pdof{G}{q^i}\left(\pdof{H}{p_i}\right)\\
        &=\sum_i \pdof{H}{p_i}\pdof{G}{q^i} - \pdof{H}{q^i}\pdof{G}{p_i}\\
        &=\vec{H}(G).
    \end{align*}
    Therefore,
    \begin{align*}
            &G(\lambda(t)) \equiv \text{constant}\\
            &\Leftrightarrow \frac{d}{dt}G(\lambda(t))\equiv 0\\
            &\Leftrightarrow \vec{H}(G) = 0\\
            &\Leftrightarrow \{H,G\} = 0
    \end{align*}
\end{proof}
\begin{cor}
    $H$ is the integral of $\dot \lambda = \vec H(\lambda)$
\end{cor}
\begin{proof}
    $\{H,H\} = 0$ by antisymmetry.
\end{proof}
\begin{thm}[Noether's Theorem in the Hamiltonian setting]
    Assume that the mechanical system is defined by a Lagrangian $L:TM\to \R$ and $H$ is the corresponding Hamiltonian (via the Legendre transform). Let $X$ be a vector field on $M$ which generates a 1-parameter group of symmetries of $L$, $\vhi^s$. Then, for $q\in M, p\in T_q^*M$,
    \[H_X(p,q) = p(X(q)),\]
    is an integral of the system $\dot \lambda = \vec H(\lambda)$.
\end{thm}
\begin{proof}
    Recall from the previous lecture that a curve $q(t)\in M$ is a solution to the Lagrangian system if and only if, for $p(t) = \pdof{L}{\dot q}\big|_{(q(t),\dot q(t))}$, then the curve $\lambda(t) = (p(t),q(t))$ is a trajectory for the Hamiltonian system. We also recall that by the original Noether Theorem,
    \[I(q,\dot q) = \pdof{L}{\dot q^i}X^i\]
    is a first integral for the Lagrangian system. Therefore, letting $\lambda(t) = (p(t),q(t))$ be a trajectory, we have that 
    \begin{align*}
        H_X(\lambda(t)) &= p(t)\left(X_{q(t)}\right)\\
        &=\pdof{L}{\dot q}\Big|_{(q(t),\dot q(t))}X_{q(t)}\\
        &=\pdof{L}{\dot q}\Big|_{(q(t),\dot q(t))}\pdof{\vhi^s}{s}\Big|_{s=0}
        &=I(q(t),\dot q(t))
    \end{align*}
    This quantity is constant along $(q(t),\dot q(t))$, so $H_x(\lambda(t))$ is constant.
\end{proof}
\begin{rk}
    $H_X$ is called the quasi-impulse of $X$ (by the analogy that $H_{\pd{q^i}} = p_i$). $H_X$ is linear on each fiber, and vice-versa if $H\in C^\infty(T^*M)$ is linear on each fiber then there exists $X\in \mathcal{X}(M)$ such that $H=H_X$
\end{rk}
\begin{rk}
    If $X,Y\in \mathcal{X}(M)$, then $\{H_X,H_Y\} = H_{[X,Y]}$. Therefore, the algebra of infinitesimal symmetries of the Lagrangian (w.r.t Poisson Bracket) is isomorphic to the algebra of integrals of the Hamiltonian system $\dot \lambda = \vec H(\lambda)$ which is linear on fibers of $T^*M$.
\end{rk}
\begin{ex}
    Integrals in $\R^3$
\end{ex}
Let $M=\R^3\sim \vec F$. Let
\[L = \frac{m|\vec r|^2}{2} - U(|\vec r|)\]
This is called the central field. Then, the group of symmetries of $L$ is $SO(3)$, so 
\[H_i = \ang{\vec r\times m\dot{\vec r},\vec{e}} = \ang{m\dot{\vec r},\vec{e}\times\vec r}\]
are integrals of E-L (identifying the vector $\vec e\times \vec r$ with the generator $Y_{\vec{e}}\in\mathfrak{so}(3)$. Also, $H_i(p,\vec r) = p(\vec e\times \vec {r})$ are integrals of $\dot\lambda = \vec H(\lambda)$. (Recall that $H = \frac{\|p\|^2}{2m} + U(\vec r)$). It can be shown that
\[\begin{array}{ccc}
     (Y_1,Y_2) = Y_3 & & \{H_1,H_2\} = H_3\\
     (Y_2,Y_3) = Y_1 &\Leftrightarrow & \{H_2,H_3\} = H_1\\
     (Y_3,Y_1) = Y_2 & & \{H_3,H_1\} = H_2\\
\end{array}.\]
On the other hand, if $G = H_1^2 + H_2^2 + H_3^2$, then we claim that $\{G,H_i\} = 0$. For example,
\begin{align*}
    \{G,H_3\} &= \{H_1^2 + H_2^2 + H_3^2, H_3\}\\
    &=2H_1\{H_1,H_3\} + 2H_2\{H_2,H_3\} + 2H_3\{H_3,H_3\}\\
    &=2H_1(-H_2) + 2H_2(H_1) + 2H_3(0)\\
    &=0
\end{align*}
Hence, $(H,G,H_i)$ are integrals which pairwise commute.
\begin{rk}
    The construction of Hamiltonian vector fields and Poisson brackets can be done exactly the same way for any Symplectic manifold $(N,\w)$.
\end{rk}
All symplectic manifolds of the same dimension are locally equivalent, i.e. if $(N_1,\w_1)$ and $(N_2,\w_2)$ are 2 symplectic manifolds with $\dim N_1 = \dim N_2$, then for all $\lambda_1\in N_1$ and $\lambda_2\in N_2$, there exists a neighborhood $U_1$ and $U_2$ of each point and a diffeomorphism $\vhi:U_1\to U_2$ such that $\vhi(\lambda_1) = \lambda_2$ and that $\vhi^*\w_2 = \w_1$.
\begin{thm}[Darboux's Theorem]
    If $(N,\w)$ is a a symplectic manifold of dimension $2n$, then around every point $\lambda$, there exists a coordinate system $(p_1,\dots, p_n,q^1,\dots, q^n)$ such that $\w = \sum_i dp_i\wedge dq^i$.
\end{thm}
WILL GO THROUGH THE PROOF LATER
\end{document}
