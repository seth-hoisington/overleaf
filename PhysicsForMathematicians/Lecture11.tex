\documentclass{article}
\usepackage[utf8]{inputenc}
\usepackage{amsfonts}
\usepackage{amsmath}
\usepackage{amsthm}
\usepackage{amssymb}
\usepackage{mathtools}
\usepackage{tikz}
\usepackage{quiver}
\usepackage{tikz-cd}
\usepackage{bbm}
\usepackage{graphicx}
\usepackage{enumitem}

\newcommand{\R}{\mathbb R}
\newcommand{\N}{\mathbb N}
\newcommand{\Q}{\mathbb Q}
\newcommand{\Z}{\mathbb Z}
\newcommand{\C}{\mathbb C}
\newcommand{\HH}{\mathcal H}
\newcommand{\posRcl}{{\mathbb R}_{\geq 0}}
\newcommand{\one}{\mathbbm{1}}
\newcommand{\g}{\mathfrak{g}}
\newcommand{\curlyL}{\mathcal L}

\newcommand{\eps}{\varepsilon}
\newcommand{\nl}{\newline\newline\noindent}
\newcommand{\cpt}{[0,1]}
\newcommand{\bI}{\mathbf{I}}
\newcommand{\xv}{\vec{x}}
\newcommand{\yv}{\vec{y}}
\newcommand{\al}{\alpha}
\newcommand{\be}{\beta}
\newcommand{\ga}{\gamma}
\newcommand{\de}{\delta}
\newcommand{\topo}{{T}}
\newcommand{\vhi}{\varphi}
\newcommand{\rank}{\text{rank}}
\newcommand{\sgn}{\text{sgn}}
\newcommand{\w}{\omega}
\newcommand{\pd}[1]{\frac{\partial}{\partial #1}}
\newcommand{\pdof}[2]{\frac{\partial #1}{\partial #2}}
\newcommand{\inv}[1]{#1^{-1}}
\newcommand{\bigslant}[2]{\left.\raisebox{.1em}{$#1$}\middle/\raisebox{-.15em}{$#2$}\right.}
\newcommand{\Int}{\text{Int}}
\newcommand{\im}{\text{im}\,}
\newcommand{\Hopf}{\text{Hopf}\,}
\newcommand{\bra}{\langle}
\newcommand{\ket}{\rangle}
\DeclareMathOperator{\supp}{supp}
\DeclareMathOperator{\spn}{span}
\DeclareMathOperator{\Hom}{Hom}
\DeclareMathOperator{\tr}{tr}
\DeclareMathOperator{\Div}{div}
\DeclarePairedDelimiter{\ang}{\langle}{\rangle}
\title{MATH 689 - Physics for Mathematicians}
\author{Lectures by Igor Zelenko, transcribed by Seth Hoisington}
\date{September 26, 2023}

\newtheorem{thm}{Theorem}
\newtheorem{ex}{Example}
\newtheorem{defn}{Definition}
\newtheorem{lem}{Lemma}
\newtheorem{cor}{Corollary}
\newtheorem{rk}{Remark}

\begin{document}

\maketitle

Let $S^0$ be a density of a conserved quantity on $\R^{n+1} = (t,x)$. For example, a density of energy, momentum, charge, or mass. To $S^0$, one can assign a vector-valued quantity $\bar S = (S^1,\dots, S^n)$, the current density. In the previous example, $S^0 = \rho$, $\bar S = \rho v$.
\nl
The conservation law will take the form
\[\pdof{ S^0}{t} + \sum_{i=1}^n \pdof{S^i}{x^i} = 0 \Leftrightarrow -\frac{d}{ds}\int_\Omega Sdx = \int_{\partial \Omega} \bar S\cdot \hat h dS\]
Then the $n+1$-dimensional quantity $\sigma = (S^0, S^1,\dots,S^n)$ is called the conserved current/the conserved flux, and it satisfies $\Div_{(t,x)}\sigma= 0$. If we assume that $S^i(t,x) \to 0$ as $|x| \to \infty$, sufficiently quickly, so that the integral converges, and if the "charge,"
\[Q(t_0) = \int_{\{t=t_0\}} S^0(t,x)dx,\]
then if $\sigma$ is conserved, then $Q$ is independent of $t_0$ (i.e. the charge is conserved).
\nl
Indeed, we have
\[Q(t_1) - Q(t_0)\]
FILL IN WITH PICTURE
\nl
If $n=0$, we are in the case of finite degrees of freedom. $(1)\Leftrightarrow \pdof{S^0}{t} = 0$, so $S^0$ is constant along solutions of E-L, meaning that $S^0$ is the first integral of E-L in the previous sense.
\nl
\begin{rk}
    Trivially conserved quantities
\end{rk}
let $\psi^{k\ell}$ be an arbitrary tuple of functions depending on fields and derivatives such that $\psi^{\ell k} =- \psi^{k\ell}$. Let $S^\mu = \sum_{\ell}\pd{x^\ell}\psi^{\mu\ell}$. Then $(S^\mu)_{\mu = 0}^n$ is conserved (independent of the field), in that
\begin{align*}
    \sum_{\mu}\pdof{x^\mu}S^\mu &= \sum_{\mu,\ell}\frac{\partial^2}{\partial x^\mu \partial x^\ell}\psi^{\mu\ell}\\
    &=\sum_{\mu<\ell}\frac{\partial^2}{\partial x^\mu \partial x^\ell}\psi^{\mu\ell} + \sum_{\ell<\mu}\frac{\partial^2}{\partial x^\mu \partial x^\ell}\psi^{\mu\ell}+\sum_{\mu}\frac{\partial^2}{\partial x^\mu \partial x^\mu}\psi^{\mu\mu}\\
    &=\sum_{\mu<\ell}\frac{\partial^2}{\partial x^\mu \partial x^\ell}\psi^{\mu\ell} + \sum_{\mu<\ell}\frac{\partial^2}{\partial x^\mu \partial x^\ell}\psi^{\ell\mu}+0\\
    &=\sum_{\mu<\ell}\frac{\partial^2}{\partial x^\mu \partial x^\ell}\left[\psi^{\mu\ell} + \psi^{\ell\mu}\right]\\
    &=0
\end{align*}
Where $\psi^{\mu\ell} + \psi^{\ell\mu} = 0$ and $\psi^{\mu\mu} = 0$ by antisymmetry.
\section{Energy-Momentum tensor}
Assume that the Lagrangian density is independent of $(t,x)$. Namely, $\curlyL = \curlyL(u,\partial_x u)$, where $x\in \R^{n+1}$ (includes the "time" coordinate $ct$). Recall that for the case of finite degrees of freedom, then $L=L(q,\dot q)$ is the law of conservation of energy. Using the same idea, we want
\[\sum_i\pd{x^i}\curlyL(u(x),\partial_xu(x)) = \pdof{\curlyL}{u}\pdof {u}{x^i} +\sum_k\pdof{\curlyL}{\partial(\partial_xu)}\frac{\partial^2u}{\partial x^i\partial x^k} = 
\sum_{k,i}\pd{x^k}\left(\pdof{L}{(\partial_{x^k}u)}\right)\pdof{u}{x^i}+\pdof{\curlyL}{\partial(\partial_xu)}\frac{\partial^2u}{\partial x^i\partial x^k} = \sum_k\pd{x^k}\left(\sum_i\pdof{L}{(\partial_{x^k}u)}\pdof{u}{x^i}\right).\]
Hence, 
\[\sum_k\pd{x^k}\left(\sum_i\pdof{L}{(\partial_{x^k}u)}\pdof{u}{x^i} - \delta^k_iL\right) = 0\]
Let $T^k_i = \partial_{x^i}u\pdof{L}{(\partial_{x^k}u)} - \delta^k_iL$. Then $\pd{x^k}T^k_i = 0$, so $(T^k_i)_{k=0}^n$ is a conserved current.
\end{document}