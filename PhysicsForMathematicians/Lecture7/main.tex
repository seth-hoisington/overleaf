\documentclass{article}
\usepackage[utf8]{inputenc}
\usepackage{amsfonts}
\usepackage{amsmath}
\usepackage{amsthm}
\usepackage{amssymb}
\usepackage{mathtools}
\usepackage{tikz}
\usepackage{quiver}
\usepackage{tikz-cd}
\usepackage{bbm}
\usepackage{graphicx}

\newcommand{\R}{\mathbb R}
\newcommand{\N}{\mathbb N}
\newcommand{\Q}{\mathbb Q}
\newcommand{\Z}{\mathbb Z}
\newcommand{\C}{\mathbb C}
\newcommand{\HH}{\mathcal H}
\newcommand{\posRcl}{{\mathbb R}_{\geq 0}}
\newcommand{\one}{\mathbbm{1}}
\newcommand{\g}{\mathfrak{g}}

\newcommand{\eps}{\varepsilon}
\newcommand{\nl}{\newline\newline\noindent}
\newcommand{\cpt}{[0,1]}
\newcommand{\bI}{\mathbf{I}}
\newcommand{\xv}{\vec{x}}
\newcommand{\yv}{\vec{y}}
\newcommand{\al}{\alpha}
\newcommand{\be}{\beta}
\newcommand{\ga}{\gamma}
\newcommand{\de}{\delta}
\newcommand{\topo}{{T}}
\newcommand{\vhi}{\varphi}
\newcommand{\rank}{\text{rank}}
\newcommand{\sgn}{\text{sgn}}
\newcommand{\w}{\omega}
\newcommand{\pd}[1]{\frac{\partial}{\partial #1}}
\newcommand{\pdof}[2]{\frac{\partial #1}{\partial #2}}
\newcommand{\inv}[1]{#1^{-1}}
\newcommand{\bigslant}[2]{\left.\raisebox{.1em}{$#1$}\middle/\raisebox{-.15em}{$#2$}\right.}
\newcommand{\Int}{\text{Int}}
\newcommand{\im}{\text{im}\,}
\newcommand{\Hopf}{\text{Hopf}\,}
\newcommand{\bra}{\langle}
\newcommand{\ket}{\rangle}
\DeclareMathOperator{\supp}{supp}
\DeclareMathOperator{\spn}{span}
\DeclareMathOperator{\Hom}{Hom}
\DeclareMathOperator{\tr}{tr}
\DeclarePairedDelimiter{\ang}{\langle}{\rangle}
\title{MATH 689 - Physics for Mathematicians}
\author{Lectures by Igor Zelenko, transcribed by Seth Hoisington}
\date{Sepember 5, 2023}

\newtheorem{thm}{Theorem}
\newtheorem{ex}{Example}
\newtheorem{defn}{Definition}
\newtheorem{lem}{Lemma}
\newtheorem{cor}{Corollary}
\newtheorem{rk}{Remark}

\begin{document}

\maketitle
\section*{Last Time:}
$\{F_i,F_j\} = 0$, $\{\vhi_i,\vhi_j\} = 0$, $\{F_i,\vhi_j\} = \delta_{ij}$.

Let $a_{ij} = \{F_{ij},\vhi_{ij}\}$

$N$ independent of $\sigma$. $\det(a_{ij}) \neq 0$.

$\{F_m, \{\vhi_i,\vhi_j\}\}$ depends on $F$ only.


LONG PROOF, DIDNT FOLLOW, GO BACK THROUGH LATER

We have $(I,\psi)$ symplectic, also called action-angle. Then $\dot\lambda = \vec H(\lambda)\Leftrightarrow H(I), \dot I = 0 = -\pdof{H}{\psi} \& \dot\psi = \w(I) = \pdof{H}{I}$
\section*{LFDAF Method of generating functions to construct action-angle variables}
Consider the action
\[S(q,t) = \min\left\{\int_{t_0}^tL(t,q(t),\dot q(t))dt\,\big|\,\text{$q(t)$ a trajectory}, q(t_0) = q_0,q(t_1) = q_1\right\}\]
Then $\pdof{S}{q} + H\left(\pdof{S}{q},q,t\right) = 0$ where $H$ is the Hamiltonian corresponding to the Lagrangian. (see Arnold, p. 254).
\nl
There is a simpler way to find action-angle variables. If $(I,\vhi)$ are symplectic coordinates, then the form $pdq - Id\vhi$ is closed because 
\[d(pdq - Id\vhi) = dp\wedge dq - dI\wedge d\vhi = 0\]
If $\gamma_k$ is a generator of $\pi_1(\mathbb{T}^n)$, then
$\int_{\gamma_k}pdq - Id\vhi$ is independent of representative in $\pi_1(\mathbb{T}^n)$, so we let
\begin{align*}
    \tilde C_{k^i} &= \int_{\gamma_k}pdq - Id\vhi\\
    &=\int_{\gamma_k}pdq - I_k\intd\vhi_k\\
    &\Rightarrow
\end{align*}
\end{document}