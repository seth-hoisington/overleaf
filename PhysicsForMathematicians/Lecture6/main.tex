\documentclass{article}
\usepackage[utf8]{inputenc}
\usepackage{amsfonts}
\usepackage{amsmath}
\usepackage{amsthm}
\usepackage{amssymb}
\usepackage{mathtools}
\usepackage{tikz}
\usepackage{quiver}
\usepackage{tikz-cd}
\usepackage{bbm}
\usepackage{graphicx}

\newcommand{\R}{\mathbb R}
\newcommand{\N}{\mathbb N}
\newcommand{\Q}{\mathbb Q}
\newcommand{\Z}{\mathbb Z}
\newcommand{\C}{\mathbb C}
\newcommand{\HH}{\mathcal H}
\newcommand{\posRcl}{{\mathbb R}_{\geq 0}}
\newcommand{\one}{\mathbbm{1}}
\newcommand{\g}{\mathfrak{g}}

\newcommand{\eps}{\varepsilon}
\newcommand{\nl}{\newline\newline\noindent}
\newcommand{\cpt}{[0,1]}
\newcommand{\bI}{\mathbf{I}}
\newcommand{\xv}{\vec{x}}
\newcommand{\yv}{\vec{y}}
\newcommand{\al}{\alpha}
\newcommand{\be}{\beta}
\newcommand{\ga}{\gamma}
\newcommand{\de}{\delta}
\newcommand{\topo}{{T}}
\newcommand{\vhi}{\varphi}
\newcommand{\rank}{\text{rank}}
\newcommand{\sgn}{\text{sgn}}
\newcommand{\w}{\omega}
\newcommand{\pd}[1]{\frac{\partial}{\partial #1}}
\newcommand{\pdof}[2]{\frac{\partial #1}{\partial #2}}
\newcommand{\inv}[1]{#1^{-1}}
\newcommand{\bigslant}[2]{\left.\raisebox{.1em}{$#1$}\middle/\raisebox{-.15em}{$#2$}\right.}
\newcommand{\Int}{\text{Int}}
\newcommand{\im}{\text{im}\,}
\newcommand{\Hopf}{\text{Hopf}\,}
\newcommand{\bra}{\langle}
\newcommand{\ket}{\rangle}
\DeclareMathOperator{\supp}{supp}
\DeclareMathOperator{\spn}{span}
\DeclareMathOperator{\Hom}{Hom}
\DeclareMathOperator{\tr}{tr}
\DeclarePairedDelimiter{\ang}{\langle}{\rangle}
\title{MATH 689 - Physics for Mathematicians}
\author{Lectures by Igor Zelenko, transcribed by Seth Hoisington}
\date{Sepember 5, 2023}

\newtheorem{thm}{Theorem}
\newtheorem{ex}{Example}
\newtheorem{defn}{Definition}
\newtheorem{lem}{Lemma}
\newtheorem{cor}{Corollary}
\newtheorem{rk}{Remark}

\begin{document}

\maketitle

\begin{thm}
    Consider a Hamiltonian system $\dot\lambda = \vec H(\lambda)$ in $N$ ($\dim N = 2n$), admitting $n$ independent integrals $(F_1 = H,\dots,F_n)$ in involution. Then, for $f = (f_1,\dots,f_n)\in \R^n$, and define
    \[N_f = \{\lambda\in N\,|\, F_1(\lambda) = f_1,\dots, F_n(\lambda) = f_n\}.\]
    Then,
    \begin{enumerate}
        \item If $N_f$ is nonempty, connected, and compact, then $N_f$ is diffeomorphic to the $n$-dimensional torus $T^n$, and one can choose global coordinates $\vhi = (\vhi_1,\dots, \vhi^n)\mod 2\pi$ on $T^n$ such that there exists $\w\in\R^n$ such that in this coordinate system, $\dot\vhi = \w$.
        \item Moreover, in the neighborhood $\tilde N$ of $N_f$, there are symplectic coordinates $(I,\vhi)$ (i.e. $\tilde N \cong B^n\times T^n$ such that in these coordinates, 
        \[\left\{\begin{aligned}
            \dot I &= 0\\
            \dot\vhi &= \w(I)
        \end{aligned}\right.\]
        $(I,\vhi)$ are called action-angle coordinates.
    \end{enumerate}
\end{thm}
\begin{proof}
    Let $g_i^t$ be the flow on $N_f$ generated by $\vec F_i$. Then, since $N_f$ is compact, then any vector field is complete, so $g_i^t$ is defined for every $t$. Since $[\vec F_i,\vec F_j] = 0$ for all $i\neq j$, then the flows commute.
    MORE DETAILS 
    Now, let $t = (t_1,\dots,t_n)\in \R^n$. Then  $g^t = g_1^t\circ g_n^t$ defines an action of $\R^n$ on $N_f$ since $g^{s+t} = g^s\circ g^t$.
    \nl
    \textbf{Claim:} Any orbit of $g$ is open.
    \nl
    Note that this implies that $N_f = O$ for $O$ an orbit since if the $N_f\neq O$, then $N_f$ is the disjoint union of at least 2 nonempty open orbits which contradicts connectedness
    \begin{proof}
        Let $O$ be an orbit. Let $\lambda_0\in O$ and consider $g_{\lambda_0}:\R^n\to N_f$ such that $g_{\lambda_0}(t) = g^t(\lambda_0)$. In particular, $g_{\lambda_0}(0) = \lambda_0$.
    \end{proof}
    \begin{lem}
        $g_{\lambda_0}$ is a diffeomorphism of a neighborhood of $0\in\R^n$ to a neighborhood $V$ of $\lambda_0$ in $N_f$. (Note that $V\subseteq O$, which implies openness of $O$ since $\lambda_0$ is arbitrary).
    \end{lem}
    \begin{proof}
        Note that 
        \begin{align*}
            (Dg_{\lambda_0})_0\left(\pd{t^i}\right) &= 
        \end{align*}
        MORE PROOF
    \end{proof}
    Now, fix $\lambda_0$. Let
    \[\Gamma = \{t\in\R^n\,|\, g^t(\lambda_0) =\lambda_0,\]
    the stabilizing subgroup of the action $g^t$ with respect to $\lambda_0$.
    MISSING STUFF
    \begin{lem}
        Let $\Gamma$ be a discrete subgroup of $\R^n$. Then there exists $k$ linearly independent vectors $\ell_1,\dots,\ell_k\in \Gamma$ ($k\in\{0,\dots,k\}$) such that
        \[\Gamma = \{\m_1\ell_1 + \cdots+m_k\ell_k\,|\, m_i\in \Z\}.\]
    \end{lem}
    \begin{proof}
        SKIPPED
    \end{proof}
    Now, let $(\ell_1,\dots,\ell_k)$ be generators of $\Gamma$ as in Lemma 2. Then, define $p:\R^k\times \R^{n-k}\to T^k\times\R^{n-k}$ which maps
    \[p(\vhi_1,\dots,\vhi_k,y_1,\dots,y_{n-k}) = (\vhi_1\mod 2\pi,\dots,\vhi_k\mod 2\pi,y_1,\dots,y_{n-k}).\]
    Let $f_i = (0,\dots,2\pi,\dots,0)\in\R^k$ with the $2\pi$ in the $i$th coordinate. Then define $A:\R^n\to \R^n$ via $A(f_i) = \ell_i$ MORE MISING

    \nl
    Since $N_f$ is compact, then $k=n$, so $N_f\cong T^n$, and we have the following commutative diagram:
    % https://q.uiver.app/#q=WzAsNCxbMCwwLCJcXFJebiJdLFsxLDAsIlxcUl5uIl0sWzAsMSwiVF5uIl0sWzEsMSwiTl9mIl0sWzAsMSwiQSJdLFswLDIsInAiLDJdLFsyLDMsIlxcdGlsZGUgQSIsMl0sWzEsMywiZ197XFxsYW1iZGFfMH0iXV0=
    \[\begin{tikzcd}
    	{\R^n} & {\R^n} \\
    	{T^n} & {N_f}
    	\arrow["A", from=1-1, to=1-2]
    	\arrow["p"', from=1-1, to=2-1]
    	\arrow["{\tilde A}"', from=2-1, to=2-2]
    	\arrow["{g_{\lambda_0}}", from=1-2, to=2-2]
    \end{tikzcd}\]
    Assume that 
\end{proof}
\begin{rk}
    If $N_f$ is not compact, but all vector fields $\vec F_i$ are complete, then $N_f \cong T^k\times \R^{n-k}$ and 0
    \[\left\{\begin{aligned}
            \dot \vhi &= \w\\
            \dot y &= 
        \end{aligned}\right.\]
\end{rk}
If the integrals $\{F_1,\dots, F_k\}$ form a basis for a noncommutative algebra $A$ ($\{F_i,F_j\} = \sum_s c_{ij}^sF_s$, then $N_f$ is diffeomorphic to a simply connected Lie group with Lie algebra $A$ given as a quotient by a discrete subgroup.

\end{document}
