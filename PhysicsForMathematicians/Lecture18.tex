\documentclass{article}
\usepackage[utf8]{inputenc}
\usepackage{amsfonts}
\usepackage{amsmath}
\usepackage{amsthm}
\usepackage{amssymb}
\usepackage{mathtools}
\usepackage{tikz}
\usepackage{tikz-cd}
\usepackage{bbm}
\usepackage{graphicx}
\usepackage{enumitem}

\newcommand{\R}{\mathbb R}
\newcommand{\N}{\mathbb N}
\newcommand{\Q}{\mathbb Q}
\newcommand{\Z}{\mathbb Z}
\newcommand{\C}{\mathbb C}
\newcommand{\HH}{\mathcal H}
\newcommand{\posRcl}{{\mathbb R}_{\geq 0}}
\newcommand{\one}{\mathbbm{1}}
\newcommand{\g}{\mathfrak{g}}
\newcommand{\curlyL}{\mathcal L}

\newcommand{\eps}{\varepsilon}
\newcommand{\nl}{\newline\newline\noindent}
\newcommand{\cpt}{[0,1]}
\newcommand{\bI}{\mathbf{I}}
\newcommand{\xv}{\vec{x}}
\newcommand{\yv}{\vec{y}}
\newcommand{\al}{\alpha}
\newcommand{\be}{\beta}
\newcommand{\ga}{\gamma}
\newcommand{\de}{\delta}
\newcommand{\topo}{{T}}
\newcommand{\vhi}{\varphi}
\newcommand{\rank}{\text{rank}}
\newcommand{\sgn}{\text{sgn}}
\newcommand{\w}{\omega}
\newcommand{\pd}[1]{\frac{\partial}{\partial #1}}
\newcommand{\pdof}[2]{\frac{\partial #1}{\partial #2}}
\newcommand{\inv}[1]{#1^{-1}}
\newcommand{\bigslant}[2]{\left.\raisebox{.1em}{$#1$}\middle/\raisebox{-.15em}{$#2$}\right.}
\newcommand{\Int}{\text{Int}}
\newcommand{\im}{\text{im}\,}
\newcommand{\Hopf}{\text{Hopf}\,}
\newcommand{\bra}{\langle}
\newcommand{\ket}{\rangle}
\DeclareMathOperator{\supp}{supp}
\DeclareMathOperator{\spn}{span}
\DeclareMathOperator{\Hom}{Hom}
\DeclareMathOperator{\tr}{tr}
\DeclareMathOperator{\Div}{div}
\DeclareMathOperator{\curl}{curl}
\DeclarePairedDelimiter{\ang}{\langle}{\rangle}
\title{MATH 689 - Physics for Mathematicians, Lecture 11}
\author{Lectures by Igor Zelenko, transcribed by Seth Hoisington}
\date{September 28, 2023}

\newtheorem{thm}{Theorem}
\newtheorem{prop}{Proposition}
\newtheorem{ex}{Example}
\newtheorem{defn}{Definition}
\newtheorem{lem}{Lemma}
\newtheorem{cor}{Corollary}
\newtheorem{rk}{Remark}

\begin{document}

\maketitle

\section{Yang-Mills Functionals and Fields}
\subsection{A crash course in principal bundles and associated bundles}
Given a Lie group $G$ let $\pi:P\to G$ be a principal bundle with structure group $G$, or, shortly a principal $G$-bundle. Very briefly, $P$ is a fiber-bundle with the fiberwise right action of $G$ which is transitive and free.
\begin{ex}
    The bundle $F(M)$ if all frames of $TM$, with $\mathbb K = \R,\C$. More concretely, $\mathcal F(M) = \{(g,L)\,|\, q\in M, L\in \text{Iso}\,(\mathbb K^n,T_pM)\}$ with $G = GL_n(\mathbb K)$, where $\text{Iso}\,(A,B)$ is the set of linear isomorphisms from $A$ to $B$. The right action $R_a(q,L) = (q,L\circ a)$ for all $a\in GL_n(\mathbb K)$.
\end{ex}
\begin{ex}
    Bundle $O_{p,q}(M)$ of orthonormal frames of $M$ with respect to a metric of signature $(p,q)$. Then $G = O_{p,q}$ and $(q,L)\in O_{p,q}(M)$ is a point $q$ and a frame $L:\mathbb K^n\to T_pM$.
\end{ex}
\begin{ex}
    Trivial circle bundle. Let $\pi:M\times S^1\to M$. Let $G = U(1) = \{e^{i\theta}\,|\, \theta\in [0,2\pi) \}$. Then the right action $R_{e^{i\theta}}(q,z) = (q, ze^{i\theta})$.
\end{ex}
Let $\mathfrak g$ be the Lie algebra of the group $G$. Given $X\in \mathfrak g$, we can define a special vector field $\sigma(X)$ tangent to the fibers called the \textbf{fundamental vector field}. In particular, we define $\sigma(X)$ to be the generator of the action of the one-parametric subgroup $\exp(tX)$: $\sigma(X)_p = \frac{d}{dt}R_{\exp(tX)}(p)\big|_{t=0}$.
\nl
The principal connection on a principal bundle $P$ is a $\mathfrak g$-valued 1-form $\w$ on $P$ such that:
\begin{enumerate}
    \item $\w(\sigma(X)) = X$ for all $X\in \mathfrak g$
    \item $(R_a)^* = \text{Ad}\,(a^{-1})\w$, $a\in G$.
\end{enumerate}
\begin{rk}
    The map $\text{Ad}:G\to GL_n(\mathfrak{g})$ is called the \textbf{adjoint representation of $G$ on $\mathfrak g$} given as $a\mapsto d(L_a\circ R_{a^{-1}})_e$. We note that $L_a\circ R_{a^{-1}}(e) = e$, so its derivative is a map $T_eG \to T_eG$ where $T_eG = \mathfrak{g}$ by definition. Since conjugation is a Lie group isomorphism, then its derivative is a Lie algebra isomorphism. Therefore, indeed, $\text{Ad }a\in GL(\mathfrak{g})$.
\end{rk}
\nl
Equivalently, is $\w$ satisfies the first condition, then, letting $H(p) = \ker\w_p$ (the horizontal distribution), then $\w$ satisfies the second condition if and only if $(R_a)_*H = H$ for all $a\in G$. In other words, we have that $H$ is an 
\nl
The curvature form $\Omega$ of the connection $\w$ is a $\mathfrak{g}$-valued 2-form defined by $\Omega(X,Y) = d\w(X,Y) + [\w(X),\w(Y)]$ where $[\cdot,\cdot]$ is the Lie bracket on $\mathfrak{g}$.
\nl
\subsection{Vector bundles associated to a principal $G$-bundle via a representation of $G$}
Let $V$ be a vector space over $\mathbb K \in \{\R,\C\}$ and let $\rho: G\to GL(V)$ be a representation. In this context, $V$ is called a $G$-module. Then one can define a vector bundle $E = P\times_G V$ as a quotient of $P\times V$ as the set of equivalence classes under the following relation: $(p,v)\sim (R_a p,\rho(a^{-1})v)$ for an $a\in G$.
\begin{ex}
    Let $E = \mathcal{F}(M)\times_{GL_n(\mathbb K)}\mathbb K^n$. In this case, $\rho$ is the standard action of $GL_n(\mathbb K)$ over $\mathbb K^n$, ($\rho(a)w = aw$). $(q,L,w)\sim (q,L\circ a, a^{-1}w) = (q,\tilde L,\tilde w)$. Then we see that $Lw = \tilde L \tilde w$ is a well-defined element of $T_qM$ on the equivalence class. Vice versa, if, for $(q,L,w), (q,\tidle L, \tilde w)$, $\tilde L\tidle w$, then $\tilde w = \tilde L^{-1}\circ L w$. So if $a = L^{-1}\circ \tilde L$, then $(q,\tilde L, \tilde w) = (q, La, a^{-1} w)\sim (q, L,w)$, so by before, $Lw = \tilde L\tilde w$. Hence, we have an isomorphism $(q,L,w)\mapsto Lw$. Therefore, $E\cong TM$.
\end{ex}
\begin{ex}
    Assume that $\rho: U(1)\to GL_n(\C)$ such that $\rho(e^{i\theta}) = e^{i\theta}I$. Then $(M\times S^1)\times_{U(1)}\C^n\cong M\times \C^n$ given by the isomorphism $(q,e^{i\theta},w)\mapsto (q,e^{i\theta}w)$
\end{ex}
\subsection{The affine connection on $P\times_G V$ induced by a principal connection on $P$}
A representation $\rho: G\to GL(V)$ induces a representation on the Lie algebra $\rho: \mathfrak{g}\to \mathfrak{gl}(V))$ given by $D_\rho(X) = \frac{d}{dt}\rho(\exp(tX))\big|_{t=0}$. If a principal connection on $P$ is given by a $\mathfrak{g}$-valued form $\w$ and $s$...Missing a bit
\nl
If $\tidle s$ is another local seciton of $p$ and $U$ is a common domain of $s$ and $\tilde s$, then there exists a unique $a:U\to G$ such that $\tilde s = R_a s$ and $\tilde s^*(\rho\circ \w) = \rho(a)^{-1}s^*(\rho\circ\w)\rho(a)+ \rho(a)^{-1}d\rho(a)$. This gives the same transformation as for a connection on the vector bundle associated to a local frame.
\nl
Therefore, there exists a unique affine connection $\nabla$ on $E = P\times_G V$ such that $s^*(\rho\circ \w)$ is its connection form with respect to the frame such that at $q$, it is equal to the equivalence classes of $(q,s(q),e_1),\dots,(q,s(q),e_n)$ where $(e_1,\dots, e_n)$ is a fixed basis.
\nl
some things here that I don't understand
\nl
and the curvature form of thhis connection (with respect to the chosen frame) is equal to $F= s^*\rho\circ \Omega$ where $\Omega$ is the curvature form of the connection on the principle bundle. $\tilde F = \rho(a)^{-1}F\rho(a)$.
\nl
$F_q\in \Lambda^2T_q^*M\otimes \rho(\mathfrak g)\subseteq \Lambda^2T_q^*M\otimes \mathfrak{gl}(V)$.
\subsection{Twisted differential}
Let $E$ be a vector bundle with affine connection $\nabla$, let $\Omega^p(E)$ be the space of $p$-forms with value in $E$. $w\in \Omega^p(E)$ is a section of a vector bundle with fiber $\Lambda^pT_q^*M\otimes E_q$ over $q$ where $E_q$ is a fiber over $q$. Then one can define a twisted differential $d^\nabla:\Omega^p(E)\to \Omega^{p+1}(E)$  such that if $u\in \Omega^0(E)$, then $d^\nabla u = \nabla u\in \Omega^1(E)$ and extended to higher-order forms by the antiderivative rule. Equivalently, we find
\[d^\nabla\w(X_0,\dots, X_p):=\sum_i (-1)^i\nabla_{X_i}\w(X_0,\dots,\hat{X}_i, \dots, X_p) + \sum_{i<j}(-1)^{i+j}\w([X_i,X_j],X_0,\dots, \hat{X}_i,\dots, \hat X_j,\dots X_p)\]
For usual scalar0valued forms (i.e. if $E = M\times \R)$ we et the usual coordinate-free definition of exterior differential (replace $\nabla_X$ by $D_X$
FILL IN STUFF AT THE END
\end{document}