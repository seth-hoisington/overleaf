\documentclass{article}
\usepackage[utf8]{inputenc}
\usepackage{amsfonts}
\usepackage{amsmath}
\usepackage{amsthm}
\usepackage{amssymb}
\usepackage{mathtools}
\usepackage{tikz}
\usepackage{quiver}
\usepackage{tikz-cd}
\usepackage{bbm}
\usepackage{graphicx}
\usepackage{enumitem}

\newcommand{\R}{\mathbb R}
\newcommand{\N}{\mathbb N}
\newcommand{\Q}{\mathbb Q}
\newcommand{\Z}{\mathbb Z}
\newcommand{\C}{\mathbb C}
\newcommand{\HH}{\mathcal H}
\newcommand{\posRcl}{{\mathbb R}_{\geq 0}}
\newcommand{\one}{\mathbbm{1}}
\newcommand{\g}{\mathfrak{g}}

\newcommand{\eps}{\varepsilon}
\newcommand{\nl}{\newline\newline\noindent}
\newcommand{\cpt}{[0,1]}
\newcommand{\bI}{\mathbf{I}}
\newcommand{\xv}{\vec{x}}
\newcommand{\yv}{\vec{y}}
\newcommand{\al}{\alpha}
\newcommand{\be}{\beta}
\newcommand{\ga}{\gamma}
\newcommand{\de}{\delta}
\newcommand{\topo}{{T}}
\newcommand{\vhi}{\varphi}
\newcommand{\rank}{\text{rank}}
\newcommand{\sgn}{\text{sgn}}
\newcommand{\w}{\omega}
\newcommand{\pd}[1]{\frac{\partial}{\partial #1}}
\newcommand{\pdof}[2]{\frac{\partial #1}{\partial #2}}
\newcommand{\inv}[1]{#1^{-1}}
\newcommand{\bigslant}[2]{\left.\raisebox{.1em}{$#1$}\middle/\raisebox{-.15em}{$#2$}\right.}
\newcommand{\Int}{\text{Int}}
\newcommand{\im}{\text{im}\,}
\newcommand{\Hopf}{\text{Hopf}\,}
\newcommand{\bra}{\langle}
\newcommand{\ket}{\rangle}
\DeclareMathOperator{\supp}{supp}
\DeclareMathOperator{\spn}{span}
\DeclareMathOperator{\Hom}{Hom}
\DeclareMathOperator{\tr}{tr}
\DeclarePairedDelimiter{\ang}{\langle}{\rangle}
\title{MATH 689 - Physics for Mathematicians}
\author{Lectures by Igor Zelenko, transcribed by Seth Hoisington}
\date{September 14, 2023}

\newtheorem{thm}{Theorem}
\newtheorem{ex}{Example}
\newtheorem{defn}{Definition}
\newtheorem{lem}{Lemma}
\newtheorem{cor}{Corollary}
\newtheorem{rk}{Remark}

\begin{document}

\maketitle
\textbf{REPLACE "v" WITH NU AND "p" WITH RHO IN INDICES}
\section{Recall: The Lorentz boost along $x_1$-axis}
\[\left\{\begin{aligned}
    c\tilde{t}&=\frac{\frac{y}{c}x_1 + ct}{\sqrt{1-\frac{u^2}{c^2}}}\\
    \tilde{x}_1&=\frac{x_1+\frac{u}{c} (ct)}{\sqrt{1-\frac{u^2}{c^2}}}\\
    \tilde{x}_2&= x_2\\
    \tilde{x}_3&= x_3
\end{aligned}\right.\Rightarrow \left\{\begin{aligned}
    \frac{\tilde E}{c}&=\frac{\frac{u}{c}p_1 + \frac{E}{c}}{\sqrt{1-\frac{u^2}{c^2}}}\\
    \tilde{p}_1&=\frac{p_1+\frac{u}{c} \frac{E}{c}}{\sqrt{1-\frac{u^2}{c^2}}}\\
    \tilde{p}_2&= p_2\\
    \tilde{p}_3&= p_3
\end{aligned}\right.\Rightarrow \]
\section{More about 4-vectors and 4-tensors}
We work in $\R^4$ with coordinates $(x^0 = ct,x^1, x^2, x^3)$. This space is called Minkowski space-time. The Lorentzian is defined by $\Delta\ell^2  = (\Del x^0)^2 - (\Del x^1)^2 - (\Del x^2)^2-(\Del x^3)^2$. Let 
\[(g_{\mu v})=\begin{pmatrix}
    1 & 0 & 0 &0\\
    0& -1 & 0 & 0\\
    0& 0 & -1 & 0\\
    0& 0 & 0 & -1
\end{pmatrix}\]
Then, $\Delta\ell = g_{\mu v}\Delta x^\mu \Delta x^v$. The indefinite inner product
\[a\dot\hat a = a^0\hat a^0-a^1\hat a^1- a^2\hat a^2-a^3\hat a^3\]
The linear transformation which preserves the Lorentzian inner product are called Lorentzian transformations, and are all of the form $x^\mu\mapsto \tilde x^\mu = \sum_{v=1}^3 a^\mu_v x^v$. $H$ is Lorentzian if and only if $g_{\mu v} a^\mu_p a^v_\sigma = g_{p\sigma}$ which holds if and only if the column vectors $a_\mu$
Missing stuff
\nl
The Lorentzian $\Delta\ell$ without $\star$ translations are allowed. The most general transformation between coordinates of inertial frames are affine transformation with lorentzian linear part.

\begin{defn}
    A \textbf{4-vector} is any 4-component quantity $(A_0,A_1,A_2,A_3)$ depending on an inertial frame such that when passing to another inertial frame $\tilde K$ via (4), it transforms $A^\mu\to \tilde A^\mu = \sum^3_{v=0}a^\mu_vA^v$.
\end{defn}
It can be asked what the difference is between this quantity and a 4-vector as defined linear algebraically. In this definition, care is taken to allow for the quantity labelled the 4-vector to be viewed in any inertial frame
An example is given by the relativistic 4-momentum/energy-momentum vector $\left(\frac{E}{c},p\right)$. Similarly, one can define a 4-covector and more generally the notion of tensors of type $(k,\ell)$. To do so, we let $(a_v^\mu)$ denote the inverse of $(a^\mu_v)\in O(1,3)$. Similarly, let $(g^{\lambda\rho}) = (g_{\mu\nu})^1$.
\begin{defn}
    A \textbf{4-covector} is any 4-component quantity $(A_0,A_1,A_2,A_3)$ such that $A_\mu\mapsto \tilde A_\mu = \sum_{\nu = 0}^3 a_\mu^\nu A_\nu$.
\end{defn}
\begin{rk}
    If $(A^\mu)_{\mu=0}^3$ is a. 4-vector, then $A_\mu = \sum_{\nu=0}^3g_{\mu\nu}A^\nu$ is a 4-covector. Conversely, $A^\nu = \sum_{\rho =0}^3 g^{\nu\rho}A_\rho$. Therefore,
    \[\tilde A_\mu =\sum_{\nu =0}^3 g_{\mu\nu}\tilde A^\nu = g_{\mu\nu}a^\nu_\lambda A^\lambda = g_{\mu\nu}g^{\lambda\rho}a^\nu_\lambda A_\rho = a_\mu^\rho A_p\]
\end{rk}
In fact, from the form of $g$, $(A_0,A_1,A_2,A_3) = (A^0,-A^1,A^2,-A^3)$. 

Geometrically, a covector is an element of the dual space, and the Lorentzian inner product identifies a vector space with its dual, and this identification is given by the lowering index operation: $A_\mu = \sum_{\nu=0}^3g_{\mu\nu}A^\nu$.

More generally, given nonnegative integers $k$ and $\ell$, a tuple
\[\left(A^{\mu_1,\dots,\mu_k}_{\nu_1,\dots,\nu_\ell}\right)_{\begin{aligned}
    &0\leq \mu_i\leq 3, &1\leq i\leq k\\
    &0\leq \nu_i\leq 3, &1\leq j\leq \ell
\end{aligned}}\]
Is called a $(k,\ell)$-tensor if under the transformation $x_\mu\mapsto \sum_{\nu = 0}^3 a^\mu_\nu x_\nu + b^\mu$ between two inertial frames, it transforms as follows:
\[WAY TOO MUCH TO WRITE RN\]
\begin{ex}
    Relativistic Angular Momentum
\end{ex}
The Lagrangian for relativistic a free particle is $L = -mc^2\sqrt{1-\frac{v^2}{c^2}}$ is Lorentz-invariant.
\nl
We can use Problem 2 of Homework 1 to produce 6 independent first integrals $(6 = \dim SO(1,3))$. The generators of rotations $\rightarrow$ 3 components of angular momentum. The generators of Lorentzian $\rightarrow$ 3 additional first integrals which are components of $\vec p t - \frac{E}{c^2}\vec x$ which gives us, up to a multiplication by $c$: $(ct,\vec x)\wedge (\frac{E}{c},\vec p)\in \bigwedge^2V$ (it is a $(2,0)$-vector).
\end{document}