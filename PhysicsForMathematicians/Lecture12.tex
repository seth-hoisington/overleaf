\documentclass{article}
\usepackage[utf8]{inputenc}
\usepackage{amsfonts}
\usepackage{amsmath}
\usepackage{amsthm}
\usepackage{amssymb}
\usepackage{mathtools}
\usepackage{tikz}
\usepackage{quiver}
\usepackage{tikz-cd}
\usepackage{bbm}
\usepackage{graphicx}
\usepackage{enumitem}

\newcommand{\R}{\mathbb R}
\newcommand{\N}{\mathbb N}
\newcommand{\Q}{\mathbb Q}
\newcommand{\Z}{\mathbb Z}
\newcommand{\C}{\mathbb C}
\newcommand{\HH}{\mathcal H}
\newcommand{\posRcl}{{\mathbb R}_{\geq 0}}
\newcommand{\one}{\mathbbm{1}}
\newcommand{\g}{\mathfrak{g}}
\newcommand{\curlyL}{\mathcal L}

\newcommand{\eps}{\varepsilon}
\newcommand{\nl}{\newline\newline\noindent}
\newcommand{\cpt}{[0,1]}
\newcommand{\bI}{\mathbf{I}}
\newcommand{\xv}{\vec{x}}
\newcommand{\yv}{\vec{y}}
\newcommand{\al}{\alpha}
\newcommand{\be}{\beta}
\newcommand{\ga}{\gamma}
\newcommand{\de}{\delta}
\newcommand{\topo}{{T}}
\newcommand{\vhi}{\varphi}
\newcommand{\rank}{\text{rank}}
\newcommand{\sgn}{\text{sgn}}
\newcommand{\w}{\omega}
\newcommand{\pd}[1]{\frac{\partial}{\partial #1}}
\newcommand{\pdof}[2]{\frac{\partial #1}{\partial #2}}
\newcommand{\inv}[1]{#1^{-1}}
\newcommand{\bigslant}[2]{\left.\raisebox{.1em}{$#1$}\middle/\raisebox{-.15em}{$#2$}\right.}
\newcommand{\Int}{\text{Int}}
\newcommand{\im}{\text{im}\,}
\newcommand{\Hopf}{\text{Hopf}\,}
\newcommand{\bra}{\langle}
\newcommand{\ket}{\rangle}
\DeclareMathOperator{\supp}{supp}
\DeclareMathOperator{\spn}{span}
\DeclareMathOperator{\Hom}{Hom}
\DeclareMathOperator{\tr}{tr}
\DeclareMathOperator{\Div}{div}
\DeclarePairedDelimiter{\ang}{\langle}{\rangle}
\title{MATH 689 - Physics for Mathematicians, Lecture 11}
\author{Lectures by Igor Zelenko, transcribed by Seth Hoisington}
\date{September 28, 2023}

\newtheorem{thm}{Theorem}
\newtheorem{ex}{Example}
\newtheorem{defn}{Definition}
\newtheorem{lem}{Lemma}
\newtheorem{cor}{Corollary}
\newtheorem{rk}{Remark}

\begin{document}

\maketitle

\section{N\"oether's Theorem for classical fields}
Let $\curlyL(x,u,\partial_x u)$ be a Lagrangian density for $x\in \R^n$ (or more generally, an $n$-dimensional manifold.
\nl
Let $\phi$ be a diffeomorphism of the space $(x,u)$ of the following "triangular form": $\phi:(x,u)\mapsto (\vhi(x),F(x,u))$ Let $\bar x =\vhi(x)$ and $\bar u = F(x,u)$. Goemetrically, we have a fiber bundle $E\to M$, and fields are the sections of this bundle. $\phi: E\to E$ is a bundle map (sends fibers to fibers).
\nl
$x$ are coordinates on the base space $M$ and $u$ are the coordinates on the fibers. Therefore, the diffeomorphism $\vhi$ can be viewed as a change of coordinates on the base space and when $x$ is fixed ($\vhi(x) = x$), $F(x,u)$ is a change of coordinates on the fiber.
\nl
Then, we define the action on a region $\Omega$ in $x$-variables by
\[A_\Omega(u) = \int_\Omega \curlyL(x,u,\partial_x u) dx\]
In our new coordinates $(\tilde x,\tilde u)$ on the region $\tilde\Omega = \vhi(\Omega)$, we have
\[\tilde A_{\tilde \Omega}(\tilde u) = \int_{\tilde\Omega}d\tilde x.\]
A first guess is to the define a symmetry as a diffeomorphism $\phi$ such that $\tilde A_{\tilde \Oemga}(\tilde u) = A_\Omega(u)$. But in fact, this is too restrictive. 

Consider 
\[\delta A_{\Omega} = A_{\tilde \Omega}(\tilde u) - A_\Omega(u) = \int_\Omega \curlyL(\tilde x,\tilde u,\partial_{\tilde x}\tilde u)\det\frac{\partial \tilde x}{\partial x} - \curlyL(\tilde x,\tilde u,\partial_{\tilde x}\tilde u) dx\]
MISSING STUFF
\nl
In our previous considerations, we assumed that $\Lambda = 0$. Keeping this in mind, we assume that wåe have a one-parametric family of diffeomorphisms $\phi^s$ of the triangular form: $\phi^s:(x,u)\mapsto (x^s,u^s):=(\vhi^s(x),F^s(x,u))$. Let $X_M(x) = \frac{d}{ds}\vhi^s(x)\big|_{s=0}$ and let $X_v(x,u) = \frac{d}{ds}F^s(\vhi^s(x),u)\big|_{s=0} = \frac{d}{ds}u^s(x^s)\big|_{s=0}$. Then we want to require
\[\frac{d}{ds}\left(\curlyL(x^s, u^s,\partial_{x^s}u^s)\det\pdof{x^s}{x} - \curlyL(x,u,\partial_xu)\right)\big|s=0 = \Div_x\Lambda\]
MISSING STUFF HERE TOO
\begin{theorem}{N\"oether's Theorem for field theory}
    If we have a symmetry $\phi$ for the Lagrangian $\curlyL$ with the function $\Lambda$, then we have a conserved current
    \[J^k = -\left(\pdof{\curlyL}{\partial_{x^k}u_j}(\partial_iu^j) - \delta^k_iL\right)X^i_M + \pdof{\curlyL}{\partial_{x^m}u^s}X^s_v - \Lambda^k\]
\end{theorem}
\begin{ex}
    The space-time translation gives the corresponding current from the energy-momentum density.
\end{ex}
\begin{ex}
    Assume that $E=\R^4\times \R^N$. The fields are the sections of this bundle. Let $\Sigma: O(1,3)\to GL(\R^N)$ be a representation of $O(1,3)$ on $\R^N$ and $G = O(1,3)$ acts on $E$.
\end{ex}
Then $(x,u)\mapsto (\tilde x,\tilde u) = (Ax,\Sigma(A)u(A^{-1}x))$. Then $\tilde u(\tilde x) = \Sigma(A)u(A^{-1}Ax) = \Sigma(A)u(x)$, so $\tilde u(\tilde x) = \Sigma(A)u(x)$.
\nl
We say that the Lagrangian density $\curlyL$ is (strongly) Lorentz-invariant if for every $A\in SO(1,3)$, $\curlyL(\tilde x,\tilde u,\partial_{\tilde x}\tilde u) = \curlyL(x,u,\partial_x u)$. Note that this is teh same as the symmetry condition with $\Lambda = 0$ since $\det\pdof{\tidle x}{x} = \det A = 1$. Let $\phi$ be the 1-parametric subgroup $\{e^{sB}\,|\,s\in\R\}$ of $SO(1,3)$ which is generated by the Lie algebra element $B\in \mathfrak{so}(1,3)$. Note that $B = (b^\mu_\nu)\in \mathfrak{so}(1,3)$ iff $b_{\mu\nu} = -b_{\nu\mu}$. Indeed, $A(t)\in SO(1,3)$ iff 
\[g_{\mu\nu}a^\mu_\rho(t)a^\nu_\sigma(t) = g_{\rho\sigma},\]
where $(g_{\rho\sigma} = \begin{pmatrix}
    1 & 0 & 0 & 0 \\
    0 & -1 & 0 & 0 \\
    0 & 0 & -1 & 0\\
    0 & 0 & 0 & -1
\end{pmatrix}$
Differentiate wrt $t$ at $t=0$ and use the $A(0) = \Gamma$???
Given $\mu\new\nu$, let $B^{\mu\nu}$ be the matrix in $\mathfrak{so}(1,3)$ such that MISSING STUFF
\end{document}