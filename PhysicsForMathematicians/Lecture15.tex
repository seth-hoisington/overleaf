\documentclass{article}
\usepackage[utf8]{inputenc}
\usepackage{amsfonts}
\usepackage{amsmath}
\usepackage{amsthm}
\usepackage{amssymb}
\usepackage{mathtools}
\usepackage{tikz}
\usepackage{quiver}
\usepackage{tikz-cd}
\usepackage{bbm}
\usepackage{graphicx}
\usepackage{enumitem}

\newcommand{\R}{\mathbb R}
\newcommand{\N}{\mathbb N}
\newcommand{\Q}{\mathbb Q}
\newcommand{\Z}{\mathbb Z}
\newcommand{\C}{\mathbb C}
\newcommand{\HH}{\mathcal H}
\newcommand{\posRcl}{{\mathbb R}_{\geq 0}}
\newcommand{\one}{\mathbbm{1}}
\newcommand{\g}{\mathfrak{g}}
\newcommand{\curlyL}{\mathcal L}

\newcommand{\eps}{\varepsilon}
\newcommand{\nl}{\newline\newline\noindent}
\newcommand{\cpt}{[0,1]}
\newcommand{\bI}{\mathbf{I}}
\newcommand{\xv}{\vec{x}}
\newcommand{\yv}{\vec{y}}
\newcommand{\al}{\alpha}
\newcommand{\be}{\beta}
\newcommand{\ga}{\gamma}
\newcommand{\de}{\delta}
\newcommand{\topo}{{T}}
\newcommand{\vhi}{\varphi}
\newcommand{\rank}{\text{rank}}
\newcommand{\sgn}{\text{sgn}}
\newcommand{\w}{\omega}
\newcommand{\pd}[1]{\frac{\partial}{\partial #1}}
\newcommand{\pdof}[2]{\frac{\partial #1}{\partial #2}}
\newcommand{\inv}[1]{#1^{-1}}
\newcommand{\bigslant}[2]{\left.\raisebox{.1em}{$#1$}\middle/\raisebox{-.15em}{$#2$}\right.}
\newcommand{\Int}{\text{Int}}
\newcommand{\im}{\text{im}\,}
\newcommand{\Hopf}{\text{Hopf}\,}
\newcommand{\bra}{\langle}
\newcommand{\ket}{\rangle}
\DeclareMathOperator{\supp}{supp}
\DeclareMathOperator{\spn}{span}
\DeclareMathOperator{\Hom}{Hom}
\DeclareMathOperator{\tr}{tr}
\DeclareMathOperator{\Div}{div}
\DeclareMathOperator{\curl}{curl}
\DeclarePairedDelimiter{\ang}{\langle}{\rangle}
\title{MATH 689 - Physics for Mathematicians, Lecture 11}
\author{Lectures by Igor Zelenko, transcribed by Seth Hoisington}
\date{September 28, 2023}

\newtheorem{thm}{Theorem}
\newtheorem{prop}{Proposition}
\newtheorem{ex}{Example}
\newtheorem{defn}{Definition}
\newtheorem{lem}{Lemma}
\newtheorem{cor}{Corollary}
\newtheorem{rk}{Remark}

\begin{document}

\maketitle

\section{Electromagetic Potential}
Recall from the previous lecture that $\Div_x\vec B = 0$ implies that there exists some $\vec A$ such that $\curl_x\vec A = \vec B$. Similarly, since $\curl_x(\vec E + \frac{1}{c}\pd{t}\vec A) = 0$, we know that there exists some function $\vhi$ such that $-\nabla_x f= \vec E + \frac{1}{c}\pd{t}\vec A$.
\nl
Note that $\vec A$ is defined up to transformation by some conservative vector field. Indeed, for a function $f$, we consider $\vec{\tilde A} = \vec A - \nabla_x f$. Then, since $\curl\circ\nabla  = 0$, we have that $\curl\vec{\tilde A} = \curl{\vec A} = \vec B$. Similarly, we must also transform $\vhi$. Let $\tilde \vhi = \vhi - \frac{1}{c}\pd{t}f$. Then,
\begin{align*}
    nrjwkngfjrwknf
\end{align*}
We usually work with the 4-vector electromagnetic potential, which is written in two forms:
\begin{align*}
    (A^\mu)_{\mu = 0}^3 &= (\vhi, \vec A) \\ (A_\mu)_{\mu = 0}^3 &= (\vhi, -\vec A)
\end{align*}
Let $\partial_\mu = \pd{x^\mu}$ and $\partial^\mu = g^{\mu\nu}\partial_\nu$. For $x_0 = ct$, we have
\begin{align*}
    \partial_\mu f &= \left\{\frac{1}{c}\pdof{f}{t},\pdof{f}{x^1},\pdof{f}{x^2},\pdof{f}{x^3}\right\}
    \partial^\mu f &= \left\{\frac{1}{c}\pdof{f}{t},-\pdof{f}{x^1},-\pdof{f}{x^2},-\pdof{f}{x^3}\right\}
\end{align*}
A choice of electromagnetic potential is called a choice of gauge. For example, a classical gauge is the  \textbf{Lorenz gauge}(different person than Lorentz!), which satisfies
\[\frac{1}{c}\pd{t}\vhi + \Div_x\vec{A} = 0 \Leftrightarrow \partial_\mu\vec A^\mu = 0.\]
To find a Lorenz gauge from an arbitrary gauge $(\vhi,\vec A)$, we must find a function $f$ such that
\begin{align*}
    0 &= \frac{1}{c}\pdof{\tilde\vhi}{t} + \Div_x\vec {\tilde A}\\
    &=\frac{1}{c}\pdof{\vhi}{t} + \frac{1}{c^2}\frac{\partial^2}{\partial t^2}f + \Div_x\vec {\tilde A} - \Div_x\nabla_x f\\
\end{align*}
MISSING STUFF
\[F = (E^1dx^1+E^2dx^2 + E^3dx^3)\wedge x^0 + B^1dx^2\wedge dx^3 + B^2 dx^3\wedge dx^1 + B^3 dx^1\wedge dx^2\]
Then, under gauge transformation, 
\[-dA=F\Leftrightarrow \left\{\begin{aligned}
    \curl_x\vec A &= \vec B\\
    -\nabla_x\vhi &= \vec E + \frac{1}{c}\pd{t}\vec A
\end{aligned}\right.\]
Hence, we have $dF = 0$. MISSING CALCULATIONS
\begin{prop}
    $\Leftrightarrow *d*F = 4\pi j$.
\end{prop}
% \begin{rk}
%     The operator $*d*$ is called the \textbf{codifferential}, and is often denoted $d^*$.
% \end{rk}
\begin{proof}
    \begin{align*}
        *F &= E^1dx^2\wedge dx^3 - E^2dx^1\wedge dx^3 + E^3dx^1\wedge dx^2 + B^1dx^0\wedge dx^1 + B^2dx^0\wedge dx^2 + B^3dx^0\wedge dx^3
    \end{align*}
    Then,
    \begin{align*}
        d(*F) &= \Div_x E dx^1\wedge dx^2\wedge dx^3 + \left(\pdof{E^1}{x^0} + \left(\pdof{B^2}{x^3} - \pdof{B^3}{x^2}\right)\right)dx^0\wedge dx^2\wedge dx^3\\
        &\quad-\left(\pdof{E^2}{x^0} + \left(\pdof{B^1}{x^3} - \pdof{B^3}{x^1}\right)\right)dx^0\wedge dx^1\wedge dx^3+\left(\pdof{E^3}{x^0} + \left(\pdof{B^1}{x^2} - \pdof{B^2}{x^1}\right)\right)dx^0\wedge dx^1\wedge dx^2
    \end{align*}
    Write $j = \sum_{\mu=0}^3 j_\mu dx^\mu = \rho dx^0 - \frac{1}{c}\left(j^1dx^1 + j^2 dx^2 + j^3 dx^3\right)$. Then,
    \begin{align*}
        *d*F &= \Div_x E dx^0 + \left(\frac{1}{c}\pdof{E^1}{t} - (\curl_x\vec B)^1\right)dx^1\\
        &\quad +\left(\frac{1}{c}\pdof{E^2}{t} - (\curl_x\vec B)^2\right)dx^2 + \left(\frac{1}{c}\pdof{E^3}{t} - (\curl_x\vec B)^3\right)dx^3\\
        &=4\pi j
    \end{align*}
\end{proof}
\begin{rk}
    Assume that we are in a $(1,n-1)$-signature manifold. Then, let $\al$ be a $(p-1)$-form, and let $\be$ be a $p$-form. Let $Vol_g$ be the volume form. Then,
    \[\int_\Omega g(d\al, \be)Vol_g = (1)^{p(n-p)}\int_\Omega g(\al, *d*\be)Vol_g - \int_{\partial\Omega}\al\wedge *\be\]
\end{rk}
\begin{rk}
    The formal adjoint of $d$ is $d^* = (-1)^{p(n-p}*d*$. For $p=2$, $d^* = *d*$.
\end{rk}
Thus, the above calculation gives $d^*F =0$. Recall that $*$ is defined by 
\[Vol_g(v\wedge *w) = g(v,w)\]
for all $p$-multivectors $v,w\in \Lambda^pV$. In $(1,n-1)$-signature, $*Vol_g = -1$ and $*1 = -Vol_g$. Also, $**\be = (-1)^{(p-1)(n-p+1)}\be$. Thus, $v\wedge *w = -g(v,w)Vol_g$. 
\end{document}