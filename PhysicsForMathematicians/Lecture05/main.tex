\documentclass{article}
\usepackage[utf8]{inputenc}
\usepackage{amsfonts}
\usepackage{amsmath}
\usepackage{amsthm}
\usepackage{amssymb}
\usepackage{mathtools}
\usepackage{tikz}
\usepackage{quiver}
\usepackage{tikz-cd}
\usepackage{bbm}
\usepackage{graphicx}

\newcommand{\R}{\mathbb R}
\newcommand{\N}{\mathbb N}
\newcommand{\Q}{\mathbb Q}
\newcommand{\Z}{\mathbb Z}
\newcommand{\C}{\mathbb C}
\newcommand{\HH}{\mathcal H}
\newcommand{\posRcl}{{\mathbb R}_{\geq 0}}
\newcommand{\one}{\mathbbm{1}}
\newcommand{\g}{\mathfrak{g}}
\newcommand{\Ell}{\mathcal L}

\newcommand{\eps}{\varepsilon}
\newcommand{\nl}{\newline\newline\noindent}
\newcommand{\cpt}{[0,1]}
\newcommand{\bI}{\mathbf{I}}
\newcommand{\xv}{\vec{x}}
\newcommand{\yv}{\vec{y}}
\newcommand{\al}{\alpha}
\newcommand{\be}{\beta}
\newcommand{\ga}{\gamma}
\newcommand{\de}{\delta}
\newcommand{\topo}{{T}}
\newcommand{\vhi}{\varphi}
\newcommand{\rank}{\text{rank}}
\newcommand{\sgn}{\text{sgn}}
\newcommand{\w}{\omega}
\newcommand{\pd}[1]{\frac{\partial}{\partial #1}}
\newcommand{\pdof}[2]{\frac{\partial #1}{\partial #2}}
\newcommand{\inv}[1]{#1^{-1}}
\newcommand{\bigslant}[2]{\left.\raisebox{.1em}{$#1$}\middle/\raisebox{-.15em}{$#2$}\right.}
\newcommand{\Int}{\text{Int}}
\newcommand{\im}{\text{im}\,}
\newcommand{\Hopf}{\text{Hopf}\,}
\newcommand{\bra}{\langle}
\newcommand{\ket}{\rangle}
\DeclareMathOperator{\supp}{supp}
\DeclareMathOperator{\spn}{span}
\DeclareMathOperator{\Hom}{Hom}
\DeclareMathOperator{\tr}{tr}
\DeclarePairedDelimiter{\ang}{\langle}{\rangle}
\title{MATH 689 - Physics for Mathematicians}
\author{Lectures by Igor Zelenko, transcribed by Seth Hoisington}
\date{Sepember 5, 2023}

\newtheorem{thm}{Theorem}
\newtheorem{ex}{Example}
\newtheorem{defn}{Definition}
\newtheorem{lm}{Lemma}
\newtheorem{cor}{Corollary}
\newtheorem{rk}{Remark}

\begin{document}

\maketitle
\begin{thm}[Darboux]
    If $\sigma$ is a sympectic form on a $2n$-dimensional manifold $N$, then for all $\lambda\in N$, there exists a neighborhood $U$ of $\lambda$ and local coordinates $(p_1,\dots,p_n,q^1,\dots,q^n)$ such that
    \[\sigma = dp_1\wedge dq^1 + \cdots + dp_n\wedge dq^n\label{eqn1}\]
    (See also Arnold, p.229-232)
\end{thm}
Such coordinates are called symplectic coordinates and the transition map between symplectic coordinates is called a canonical transformation.
\begin{rk}
    
\end{rk}
From linear algebra, for every $\lambda\in N$, we can find a basis in $T_\lambda N$ such that the matrix of the bilinear form $\sigma_\lambda$ in this basis is
\[\begin{bmatrix}
    0 & I \\ -I & 0
\end{bmatrix}\]
which corresponds to the coordinate basis $\left(\pd{p_1},\dots,\pd{p_n},\pd{q^1},\dots,\pd{q^n}\right)$ in \eqref{eqn1}. In fact, such bases in $T_\lambda N$ are defined modulo the Linear Symplectic group $Sp(T_\lambda N)$. Darboux says that from closure of $\sigma$ we can choose such a basis at every point of a neighborhood such that the corresponding vector fields commute. Hence, they are coordinate bases w.r.t some coordinates.
\section{Lagrangian Submanifolds of a Symplectic Space}
\subsection{Linear algebra preliminary}
Assume that $W$ is a $2n$-dimensional vector space with a non-degenerate skew-symmetric form $\w$. (e.g. $W=T_\lambda N$, $\w = \sigma_\lambda$). Given a subspace $L\subseteq W$, let $L^\angle = \{v\in W\,|\,\w(v,z) = 0\forall z\in L\}$. Then since $\w$ is non-degenerate, we have $\dim L^\angle = \dim W - \dim L = 2n - \dim L$. $L$ is called isotropic if $L\subseteq L^\angle$. For example, if $\dim L = 1$, then for all $v,w\in L$, we have that $v=\lambda w$, so, by skew-symmetry,
\[\sigma(v,w) = \lambda \sigma(v,v) = 0.\]
If $L\subseteq W$ is isotropic, then $\dim L\leq \frac{1}{2}\dim W = n$. If $L$ is isotropic and $\dim L = n$, then $L$ is called \textbf{Lagrangian}.
\begin{ex}
    
\end{ex}
In a basis $\left(\pd{p_1},\dots,\pd{p_n},\pd{q^1},\dots,\pd{q^n}\right)$ ,  (in which $\w$ has the matrix $\begin{bmatrix}0 & I \\ -I & 0\end{bmatrix}$), we let $L_0 = \spn\left\{\pd{q^1},\dots,\pd{q^n}\right\}$ and let $L_\infty =
\spn\left\{\pd{p_1},\dots,\pd{p_n}\right\}$. Both these subspaces are Lagrangian. More generally, for any symmetric matrix $S = [S_{ij}]_{i,j=1}^n$, we have that the subspace
\[L_S = \spn\left\{\pd{q^i} + \sum_{j=1}^n S_{ij}\pd{p_j}\right\}.\]
is Lagrangian. Indeed, this follows, given that
\begin{align*}
    &\w\left(\pd{q^{i_1}} + \sum_{j=1}^n S_{i_1j}\pd{p_j},\pd{q^{i_2}} + \sum_{j=1}^n S_{i_2j}\pd{p_j}\right)\\
    &=\sum_i (dp_i\wedge dq^i)\left(\pd{q^{i_1}} + \sum_{j=1}^n S_{i_1j}\pd{p_j},\pd{q^{i_2}} + \sum_{j=1}^n S_{i_2j}\pd{p_j}\right)\\
    &=S_{i_1i_2} - S_{i_2i_1}\\
    &=0
\end{align*}
Moreover $L_S$ is Lagrangian if and only if $S$ is symmetric, and any Lagrangian $L$ transversal to $L_\infty$ is of the form $L=L_S$ for some symmetric matrix $S$.
\begin{defn}
    A submanifold $L$ of a symplectic manifold $(N,\sigma)$
 is called Lagrangian if, for every $\lambda\in L$, $T_\lambda L$ is a Lagrangian subspace of $T_\lambda N$. Equivalently, $L$ is Lagrangian if $\sigma|_L = 0$ and $\dim L = \frac{1}{2}\dim N$.
\end{defn}
\begin{ex}
    Let $N=T^*M$ with the canonical symplectic form $\sigma$. Then,
    \begin{enumerate}
        \item Every fiber of $T^*M$ is Lagrangian.
        \item Consider $M$ embedded into $T^*M$ are a graph of $v$-sections, then $M$ is defined by $dp_1 = dp_2 = \cdots = dp_n = 0$, so $\sigma_M = 0$.
    \end{enumerate}
\end{ex}
\begin{ex}
    Given a smooth function $f:M\to \R$, the graph of its differential, given by $L_f = \{((df)_q,q)\,|\, q\in M\}\subseteq T^*M$, is a Lagrangian submanifold. Then (sketching), we can show that for each $\lambda\in T^*M$,
    \[T_\lambda(T^*M) = \spn\left\{\pd{q^1} + \sum_{j=1}^n \frac{\partial^2 f}{\partial q^i\partial q^j }\pd{q^j}\right\}.\]
    Since the Hessian matrix $S=\left[\frac{\partial^2 f}{\partial q^i\partial q^j }\right]_{ij}$ is symmetric for any smooth map by Clairaut's theorem, then $L_f = L_S$. Hence, it is Lagrangian.
\end{ex}
\section{Liouville Integrability}
Let $(N,\sigma)$ be a symplectic manifold. We say that some $k$ Hamiltonians $(F_1,\dots,F_k)\in C^\infty(N)^k$  are in involution if $\{F_i,F_j\} = 0$ for all $1\leq i,j\leq k$. If $(F_1,\dots, F_k)$ are in involution, then for all $\lambda \in N$, $\spn\{(\vec F_1)_\lambda,\dots,(\vec F_k)_\lambda\}$ is an isotropic subspace of $T_\lambda N$.
\nl
We say that the Hamiltonians $(F_1,\dots,F_k)$ are independent if the vectors $\{(dF_1)_\lambda,\dots,(dF_k)_\lambda\}$ are linearly independent. By a simple linear transformation, we can conclude that $\{(\vec F_1)_\lambda,\dots,(\vec F_k)_\lambda\}$. Therefore, if a set of $k$ Hamiltonians are independent and in involution, then $k\leq n = \frac{\dim N}{2}$.
\nl
Assume that $k=n$ and $(F_1,\dots,F_n)$ are independent in involution. Then, we see that $\spn\{\vec F_1,\dots,\vec F_n\}$  is an involutive distribution over $N$ since $[\vec F_i, \vec F_j] =\overrightarrow{\{F_i,F_j\}} = \vec 0 = 0$. Therefore, the distribution is integrable by Frobenius' Theorem, so there exists an $n$-dimensional integral submanifold, which is in fact Lagrangian $(\sigma(\vec F_i,\vec F_j) = 0)$. Given this, the set $(F_1,\dots,F_n)$ which is independent in involution is called Liouville Integrable. Let $f\in \R^n$ and let $f = (f_1,\dots,f_n)$. Then, define
\[N_f = \{\lambda\in N\,|\,F_i(\lambda) = f_i,\,i=1,\dots,n\}.\]
Then if $N_f\neq \emptyset$, then $N_f$ is a Lagrangian submanifold of $N$. $N_f$ is invariant under the flow generated by $\vec F_i$ for every $i$.
\begin{rk}
    Note that if $(F_1,\dots,F_k)$ are integrals of $F_1$, then the common level set, $N_f = \{F_1=f_1,\dots,F_k=f_k\}$ for $f=(f_1,\dots,f_n)\in\R^n$, if not empty, is a codimension $k$ submanifold in $N$ and it is invariant with respect to the flow of $\vec F_1$. So, having $k$ integrals, we can reduce the degrees of freedom by $k$. The tangent space to $N_f$ at some point $\lambda\in N_f$,
    \[T_\lambda N_f = \ker (dF_1)_\lambda\cap\cdots\cap\ker(dF_k)_\lambda = \left(\spn\{\vec F_1,\dots,\vec F_k\}\right)^\angle\]
    In particular, if $(F_1,\dots,F_k)$ are in involution, then $\spn\left(\vec F_1, \dots, \vec F_k\}\right)\big|_\lambda$ is an isotropic space, so $\spn\left(\vec F_1, \dots, \vec F_k\}\right)\big|_\lambda\subseteq T_\lambda N_f$.
\end{rk}
\begin{thm}[Arnold-Liouville]
    Consider a Hamiltonian system $\dot\lambda = \vec H(\lambda)$ on $N$ admitting $n$ integrals $(F_1 - H, F_2,\dots,F_n)$ independent in involution. Let $f\in \R^n$. Then
    \begin{enumerate}
        \item If $N_f$ is nonempty, connected, and compact, then $N_f$ is diffeomorphic to an $n$-dimensional torus, and one can choose global coordinates $\vhi = (\vhi_1,\dots,\vhi_n) \mod 2\pi$ on $T^n$ such that the flow generated by $\vec H$ is conditional periodic, i.e. there exists a vector of frequencies $\w = \w(f)\in \R^n$ such that on $N_f$, we have that
        \[\dot\vhi=\w\Leftrightarrow\left\{\begin{aligned}
            \dot\vhi_1&=\w_1\\
            \vdots\\
            \dot\vhi_n&=\w_n\\
        \end{aligned}\right.\Leftrightarrow\begin{aligned}
            \vhi_1(t) & = \vhi_1(0) +\w_1 t \mod 2\pi\\
            \vdots\\
            \vhi_n(t) & = \vhi_n(0) +\w_n t \mod 2\pi
        \end{aligned}\]
        \item Moreover, in the neighborhood $\tilde N$ of $N_f$, there are symplectic coordinates $(I,\vhi)$ (i.e. $\tilde N\cong D\times T^n$) such that $\dot\lambda = \vec H(\lambda)$ is equivalent to
        \[\left\{\begin{aligned}
            \dot I&=0\\
            \dot \vhi = \w(I)
        \end{aligned}\right.\]
        Note that since $I$ is constant on $N_f$, then 
    \end{enumerate}
\end{thm}
\end{document}
