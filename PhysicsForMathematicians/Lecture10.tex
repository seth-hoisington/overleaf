\documentclass{article}
\usepackage[utf8]{inputenc}
\usepackage{amsfonts}
\usepackage{amsmath}
\usepackage{amsthm}
\usepackage{amssymb}
\usepackage{mathtools}
\usepackage{tikz}
\usepackage{quiver}
\usepackage{tikz-cd}
\usepackage{bbm}
\usepackage{graphicx}
\usepackage{enumitem}

\newcommand{\R}{\mathbb R}
\newcommand{\N}{\mathbb N}
\newcommand{\Q}{\mathbb Q}
\newcommand{\Z}{\mathbb Z}
\newcommand{\C}{\mathbb C}
\newcommand{\HH}{\mathcal H}
\newcommand{\posRcl}{{\mathbb R}_{\geq 0}}
\newcommand{\one}{\mathbbm{1}}
\newcommand{\g}{\mathfrak{g}}

\newcommand{\eps}{\varepsilon}
\newcommand{\nl}{\newline\newline\noindent}
\newcommand{\cpt}{[0,1]}
\newcommand{\bI}{\mathbf{I}}
\newcommand{\xv}{\vec{x}}
\newcommand{\yv}{\vec{y}}
\newcommand{\al}{\alpha}
\newcommand{\be}{\beta}
\newcommand{\ga}{\gamma}
\newcommand{\de}{\delta}
\newcommand{\topo}{{T}}
\newcommand{\vhi}{\varphi}
\newcommand{\rank}{\text{rank}}
\newcommand{\sgn}{\text{sgn}}
\newcommand{\w}{\omega}
\newcommand{\pd}[1]{\frac{\partial}{\partial #1}}
\newcommand{\pdof}[2]{\frac{\partial #1}{\partial #2}}
\newcommand{\inv}[1]{#1^{-1}}
\newcommand{\bigslant}[2]{\left.\raisebox{.1em}{$#1$}\middle/\raisebox{-.15em}{$#2$}\right.}
\newcommand{\Int}{\text{Int}}
\newcommand{\im}{\text{im}\,}
\newcommand{\Hopf}{\text{Hopf}\,}
\newcommand{\bra}{\langle}
\newcommand{\ket}{\rangle}
\DeclareMathOperator{\supp}{supp}
\DeclareMathOperator{\spn}{span}
\DeclareMathOperator{\Hom}{Hom}
\DeclareMathOperator{\tr}{tr}
\DeclarePairedDelimiter{\ang}{\langle}{\rangle}
\title{MATH 689 - Physics for Mathematicians}
\author{Lectures by Igor Zelenko, transcribed by Seth Hoisington}
\date{September 14, 2023}

\newtheorem{thm}{Theorem}
\newtheorem{ex}{Example}
\newtheorem{defn}{Definition}
\newtheorem{lem}{Lemma}
\newtheorem{cor}{Corollary}
\newtheorem{rk}{Remark}

\begin{document}

\maketitle
\section{Passing from the system with a finite number of particles to a continuum}

\begin{ex}
    String/elastic vibration
\end{ex}
Consider a long elastic rod subject to longitudinal vibrations. As a simplified model, we chop the rod up into small lengths of equal size and consider each piece as a particle. Then, we replace the elastic forces within the rod with ideal springs linking the particles. At rest:


Let $u_j$ be the displacement of particle $j$ from the equilibrium (state at rest) at time $t$. The mass of each particle is $\Delta m$ and the spring constant is $k$. Then, by Newton's Second Law, at the particle $j$, we have
\begin{align*}
    \Delta m \frac{d^2u_j}{dt}&= k(u_{j+1} - u_j) - k(u_j-u_{j-1})
\end{align*}
and the Lagrangian $L=T-U$ is given by
\begin{align*}
    L&=T-U\\
    &=\frac{1}{2}\sum_j \Delta m \left(\frac{d^2u_j}{dt}\right)^2 - \frac{1}{2}\sum_j k (u_{j+1} - u_j)^2\\
    &=\frac{1}{2}\sum_j \left[\frac{\Delta m}{\Delta x}\left(\frac{d^2u_j}{dt}\right)^2 - k\Delta x\left(\frac{u_{j+1} - u_j}{\Delta x}\right)^2\right]
\end{align*}
Now, we take $\Delta x \to 0$, we notice that $\frac{\Delta m}{\Delta x}$ coincides with the idea of density (which we denote $\rho$), and the quantity $k\Delta x\to Y$ (MOTIVATE THIS)
\nl
The tuple of displacement $(u_j(t))$ is replaced by a displacement function $u(x,t)$, representing the displacement of the particle $x$ at time $t$. Then we end up with the following expression:
\[\frac{1}{2}\int_a^b \rho\left(\pdof{u}{t}\right)^2 - Y\left(\pdof{u}{x}\right) dx.\]
Similarly, dividing Newton's Second Law by $\Delta x$, we have
\begin{align*}
    \frac{\Delta m}{\Delta x} \frac{d^2u_j}{dt}&= k\Delta x\frac{(u_{j+1}  -2u_j+u_{j-1})}{(\Delta x)^2}\\
    \overset{\Delta x\to 0}{\longrightarrow} \rho \frac{d^2u}{dt^2} &= Y\frac{\partial^2 u}{\partial x^2}
\end{align*}
This gives us the following action:
\[A(u) = \frac{1}{2}\int_{t_0}^{t_1}\int_a^b \rho\left(\pdof{u}{t}\right)^2 - Y\left(\pdof{u}{x}\right) dxdt.\]
\section{Classical Field Theory}
Let $\Omega\subseteq \R^{n+1} = (x^0,\dots,x^n)$ (or let $\Omega = \R^{n+1}$) and assume that $u,\nabla u$ decays sufficiently fast to zero), and let $u:\R^{n+1}\to \R$. Then,
\[\left\{\begin{aligned}
    A(u) &= \int_\Omega L(x,u,\partial_{x_i}u) dx\\
    u\big|_{\partial \Omega} &= \vhi
\end{aligned}\right.\]
In this context, we call $u$ a field. By analogy with classical mechanics of finite particle systems, we wish to find the equation for critical points of the variational problem. Fix a field $u$ and let $u^s$ be a one-parameter family of fields such that $u^0 = u$. Let $h=\pd{s}u^s\Big|_{s=0}$. Then, $h|_{\partial \Omega} = 0$. Therefore,
\begin{align*}
    \frac{d}{ds}A(u^s)\Big|_{s=0} &= \frac{d}{ds}\int_\Omega L(x,u,\partial_{x_i}u) dx\bigg|_{s=0}\\
    &=\int_\Omega \pd{s}L(x,u,\partial_{x_i}u)\Big|_{s=0} dx\\
    &=\int_\Omega L(x,u,\partial_{x_i}u)\Big|_{s=0} dx\\
\end{align*}
We end up with the Euler-Lagrange equation of the field equation. For the vibrating rod, we have
\[L = \frac{1}{2}\left(\rho\left(\pdof{u}{t}\right)^2 - Y\left(\pdof{u}{x}\right) \right)\]
Therefore, by the Newton relation,
\[-\rho \frac{d^2u}{dt^2} - Y\frac{\partial^2 u}{\partial x^2}  = 0.\]
\begin{ex}
    The Klein-Gordon Equation
\end{ex}
The most simple Lorentz-invariant Lagrangian density on a scalar field is
\[L = \frac{1}{c^2}\left(\pdof{u}{t}\right)^2 - \sum_{i=1}^3 \left(\pdof{u}{x^i}\right)^2 - m^2u^2,\]
and the corresponding E-L equation is 
\[-2m^2u - 2\frac{1}{c^2}\frac{\partial^2 u}{\partial t^2} + \sum_{i=1}^3 \left(\pdof{u}{x^i}\right)^2 = 0,\]
which (in a different form) is the Klein-Gordon Equation:
\[\underbrace{\frac{1}{c^2}\frac{\partial^2 u}{\partial t^2} - \sum_{i=1}^3 \left(\pdof{u}{x^i}\right)^2}_{\text{D'Alambertian}} + 2m^2u = 0.\]
Representing $\square u$ as the D'Alambertian applied to $u$, this is also given as
\[\square u + m^2 u = 0\]
\begin{exercise}
    Show that the Lagrangian density $L$ is Lorentz-invariant
\end{exercise}
\section{Basic on conservation laws in multivariable Calculus of variation/Field theory}
A prototype example  of a conservation law is a continuity equation for a moving fluid. Let $\rho(x,t)$ be the mass density of a fluid and let $v(x,t) = (v^1(x,t),v^2,(x,t),v^3(x,t))$ be the velocity at the point $(x^1,x^2,,x^3)$ at the time moment $t$. Then the continuity equation is the following:
\[\pdof{\rho}{t} + \pd{x^1}[\rho v^1] + \pd{x^2}[\rho v^2] + \pd{x^3}[\rho v^3] = \text{div}_{(t,x^1,x^2,x^3)}(\rho,\rho v^1,\rho v^2, \rho v^3) = 0\]
which is given in its integral form as
\[-\pd{t}\int_{\Omega}\rho(x,t)dx = -\pd{t}\int_{\Omega}\text{div}_{(x^1,x^2,x^3)}(\rho v)dx = \int_{\partial \Omega}(\rho v\dot \hat n)dS.\]
\end{document}