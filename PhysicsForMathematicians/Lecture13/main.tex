\documentclass{article}
\usepackage[utf8]{inputenc}
\usepackage{amsfonts}
\usepackage{amsmath}
\usepackage{amsthm}
\usepackage{amssymb}
\usepackage{mathtools}
\usepackage{tikz}
\usepackage{quiver}
\usepackage{tikz-cd}
\usepackage{bbm}
\usepackage{graphicx}
\usepackage{enumitem}

\newcommand{\R}{\mathbb R}
\newcommand{\N}{\mathbb N}
\newcommand{\Q}{\mathbb Q}
\newcommand{\Z}{\mathbb Z}
\newcommand{\C}{\mathbb C}
\newcommand{\HH}{\mathcal H}
\newcommand{\posRcl}{{\mathbb R}_{\geq 0}}
\newcommand{\one}{\mathbbm{1}}
\newcommand{\g}{\mathfrak{g}}
\newcommand{\curlyL}{\mathcal L}

\newcommand{\eps}{\varepsilon}
\newcommand{\nl}{\newline\newline\noindent}
\newcommand{\cpt}{[0,1]}
\newcommand{\bI}{\mathbf{I}}
\newcommand{\xv}{\vec{x}}
\newcommand{\yv}{\vec{y}}
\newcommand{\al}{\alpha}
\newcommand{\be}{\beta}
\newcommand{\ga}{\gamma}
\newcommand{\de}{\delta}
\newcommand{\topo}{{T}}
\newcommand{\vhi}{\varphi}
\newcommand{\rank}{\text{rank}}
\newcommand{\sgn}{\text{sgn}}
\newcommand{\w}{\omega}
\newcommand{\pd}[1]{\frac{\partial}{\partial #1}}
\newcommand{\pdof}[2]{\frac{\partial #1}{\partial #2}}
\newcommand{\inv}[1]{#1^{-1}}
\newcommand{\bigslant}[2]{\left.\raisebox{.1em}{$#1$}\middle/\raisebox{-.15em}{$#2$}\right.}
\newcommand{\Int}{\text{Int}}
\newcommand{\im}{\text{im}\,}
\newcommand{\Hopf}{\text{Hopf}\,}
\newcommand{\bra}{\langle}
\newcommand{\ket}{\rangle}
\DeclareMathOperator{\supp}{supp}
\DeclareMathOperator{\spn}{span}
\DeclareMathOperator{\Hom}{Hom}
\DeclareMathOperator{\tr}{tr}
\DeclareMathOperator{\Div}{div}
\DeclareMathOperator{\curl}{curl}
\DeclarePairedDelimiter{\ang}{\langle}{\rangle}
\title{MATH 689 - Physics for Mathematicians, Lecture 11}
\author{Lectures by Igor Zelenko, transcribed by Seth Hoisington}
\date{September 28, 2023}

\newtheorem{thm}{Theorem}
\newtheorem{ex}{Example}
\newtheorem{defn}{Definition}
\newtheorem{lem}{Lemma}
\newtheorem{cor}{Corollary}
\newtheorem{rk}{Remark}

\begin{document}

\maketitle

\section{N\"oether's Theorem for classical fields}
\section{Elements of Electromagnetism}
Let $\vec E(t,x)$ denote the electric field and $\vec B(t,x)$ for the magnetic field. Then, the Lorentz force on  a charge $q$ moving with velocity $\vec v$ (in Gauss units) is given by
\[\vec F = q\vec E + \frac{q}{c}\vec v\times \vec B,\]
where $c$ is some constant (not the speed of light). Classically, we have that $\vec E$ and $\vec B$ satisfy Maxwell's equations. Let $\rho$ be a charge density and $\vec j$ be the charge current. This gives us four laws:
\begin{itemize}
    \item The Gauss law for electric charge; in integral form (over a region $\Omega$):
    \[\int_{\partial \Omega} \vec E\cdot d\vec S = 4\pi \int_\Omega \rho dv\]
    In differential form:
    \[\Div \vec E = 4\pi\rho\]
    \item The Gauss law for magnetism; in integral form:
    \[\int_{\partial \Omega} \vec B\cdot d\vec S = 0\]
    In differential form:
    \[\Div \vec B = 0\]
    \item Faraday's Laws of induction; integral form (let $\partial D$ be a loop bounding a region $D$):
    \[\int_{\partial D} \vec E d\ell = -\frac{1}{c}\frac{d}{dt}\int_{D}\vec B\cdot d\vec S\]
    In differential form,
    \[\curl \vec E + \frac{1}{c}\pd{t}\vec B = 0\]
    \item Ampere's Circular Law:
    \[\int_{\partial D} \vec B\cdot d\ell = \frac{4\pi}{c}\int_{D}\vec j\cdot d\vec S + \frac{1}{c}\pd{t}\int_{D}\vec E\cdot d\vec S\]
    In differential form,
    \[\curl\vec B - \frac{1}{c}\pd{t}\vec E = \frac{4\pi}{c}\vec j\]
\end{itemize}
Now, we organize the four (differential) relations in terms of differential forms and the exterior derivative on $\R^3$. Identifying $\vec E$ and $\vec B$ as differential forms, we observe that $d \vec B  = 0$ (where $d$ is the exterior derivative). deRham Cohomology implies that there exists some $\vec A$ such that $\vec B = d\vec A = \curl \vec A$. Similarly, since 
\begin{align*}
    0 &= \curl \vec E + \frac{1}{c}\pd{t}\vec B\\
    &=\curl \vec E + \frac{1}{c}\pd{t}\curl \vec A\\
    &=\curl \left(\vec E + \frac{1}{c}\pd{t}\vec A\right)\\
    &=d\left(\vec E + \frac{1}{c}\pd{t}\vec A\right)
\end{align*}
so if we apply deRham Cohomology once again, we have that there exists $\vhi:\R^3\to \R$ such that $\nabla_x \vhi = \vec E + \frac{1}{c}\pd{t}\vec A$.
MISSING STUFF
\end{document}