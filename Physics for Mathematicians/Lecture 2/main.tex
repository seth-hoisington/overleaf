\documentclass{article}
\usepackage[utf8]{inputenc}
\usepackage{amsfonts}
\usepackage{amsmath}
\usepackage{amsthm}
\usepackage{amssymb}
\usepackage{mathtools}
\usepackage{tikz}
\usepackage{quiver}
\usepackage{tikz-cd}
\usepackage{bbm}
\usepackage{graphicx}

\newcommand{\R}{\mathbb R}
\newcommand{\N}{\mathbb N}
\newcommand{\Q}{\mathbb Q}
\newcommand{\Z}{\mathbb Z}
\newcommand{\C}{\mathbb C}
\newcommand{\HH}{\mathcal H}
\newcommand{\posRcl}{{\mathbb R}_{\geq 0}}
\newcommand{\one}{\mathbbm{1}}
\newcommand{\g}{\mathfrak{g}}

\newcommand{\eps}{\varepsilon}
\newcommand{\nl}{\newline\newline\noindent}
\newcommand{\cpt}{[0,1]}
\newcommand{\bI}{\mathbf{I}}
\newcommand{\xv}{\vec{x}}
\newcommand{\yv}{\vec{y}}
\newcommand{\al}{\alpha}
\newcommand{\be}{\beta}
\newcommand{\ga}{\gamma}
\newcommand{\de}{\delta}
\newcommand{\topo}{{T}}
\newcommand{\vhi}{\varphi}
\newcommand{\rank}{\text{rank}}
\newcommand{\sgn}{\text{sgn}}
\newcommand{\w}{\omega}
\newcommand{\pd}[1]{\frac{\partial}{\partial #1}}
\newcommand{\pdof}[2]{\frac{\partial #1}{\partial #2}}
\newcommand{\inv}[1]{#1^{-1}}
\newcommand{\bigslant}[2]{\left.\raisebox{.1em}{$#1$}\middle/\raisebox{-.15em}{$#2$}\right.}
\newcommand{\Int}{\text{Int}}
\newcommand{\im}{\text{im}\,}
\newcommand{\Hopf}{\text{Hopf}\,}
\newcommand{\bra}{\langle}
\newcommand{\ket}{\rangle}
\DeclareMathOperator{\supp}{supp}
\DeclareMathOperator{\spn}{span}
\DeclareMathOperator{\Hom}{Hom}
\DeclareMathOperator{\tr}{tr}
\DeclarePairedDelimiter{\ang}{\langle}{\rangle}
\title{MATH 689 - Physics for Mathematicians}
\author{Seth Hoisington}
\date{\today}

\newtheorem{thm}{Theorem}
\newtheorem{ex}{Example}
\newtheorem{exercise}{Exercise}
\newtheorem{defn}{Definition}
\newtheorem{rmk}{Remark}
\newtheorem{note}{Note}

\begin{document}

\maketitle

\begin{defn}
    A diffeomorphism $\vhi:M\to M$ is called a symmetry of the Lagrangian $L$ if $\vhi^*L = L$, i.e. $L(\vhi(q),\vhi_{*,q}(v)) = L(q,v)$ for all $(q,v)\in TM$.
\end{defn}

\section{Proof of Noether's Theorem}
\begin{ex}
    Shift/translation in one direction (in local coordinates)
\end{ex}
Let $H$ be described in local coordinates by $(q^1,\dots,q^n)$. If for some $i$, $L$ is independent of $q^i$, then $q^i$ is called cyclic, and $\vhi^s(q^1,\dots, q^i,\dots,q^n) = (q^1,\dots, q^i+s,\dots,q^n)$. Then $\vhi^s$ is a symmetry of $L$.
\begin{thm}[Noether's Theorem]
    If an autonomous Lagrangian $L$ admits the one-parameter group of symmetries $\vhi^s:M\to M$, then the mechanical system described by $L$ has a first integral of motion that is written in local coordinates as 
\[I(q,\dot q) = \pdof{L}{\dot q}\frac{d\vhi^s(q)}{ds}\bigg|_{s=0}\]
\end{thm}
\begin{proof}
    Let $X(q)=\frac{d\vhi^s(q)}{ds}\bigg|_{s=0}$ (the vector field generating the flow $\vhi^s$). Note that by the properties of one-parameter flows, we have that $\frac{d\phi^s}{ds}(q)=X(\vhi^s(q))$. We first note that
    \[\frac{d\dot\vhi^s(q)}{ds}\bigg|_{s=0} = \pd{s}\pd{t}\vhi^s(q)|_{s=0} = \dot X(q(t))\]
    Then we have
    \begin{align*}
        \frac{d}{dt}I &= \frac{d}{dt}\left(\pdof{L}{\dot q^i}X^i\right) \\
        &= \frac{d}{dt}\left(\pdof{L}{\dot q^i}\right)X^i+\pdof{L}{\dot q^i}\dot X^i\\
        &= \pdof{L}{q}X^i+\pdof{L}{\dot q}\dot X^i\\
        &= \pdof{L}{q^s}\bigg|_{s=0}\frac{d\vhi^s(q)}{ds}\bigg|_{s=0} + \pdof{L}{\dot q^s}\bigg|_{s=0}\frac{d\dot\vhi^s(q)}{ds}\bigg|_{s=0}\\
        &=\frac{d}{ds}L(\vhi^s(q),\dot\vhi^s(q))\big|_{s=0}
    \end{align*}
    using Euler-Lagrange equation and the chain rule.
    Since $\vhi^s$ is a symmetry of $L$, then $L(\vhi^s(q),\dot\vhi^s(q)) = L(q(t),\dot q(t))$, so $\frac{d}{ds}L(\vhi^s(q),\dot\vhi^s(q))\big|_{s=0} = 0$, giving us that $I$ is a first integral of motion.
\end{proof}
\begin{rmk}
    Coordinate-independence of $I$ in the proof of Noether's Theorem
\end{rmk}
The first IoM $I = \pdof{L}{\dot q}X$ is independent of the choice of coordinates. Fix some $q\in M$ and consider a curve $v^s\in T_qM$ such that $\frac{d}{ds} v^s\big|_{s=0} = X(q)$. Note that we are identifying $X(q)\in T_qM$ as an element of $T_v(T_qM)$. Then, we let $I(q,v) = \pd{s}L(q,v^s)\big|_{s=0}$. We see that $\pd{s}L(q,v^s)\big|_{s=0} = \pdof{L}{\dot q}(q,v)\pdof{v^s}{s}\big|_{s=0}  = \pdof{L}{\dot q}(q,v)X(q)$
\begin{ex}
    Continuation of Example 1
\end{ex}
Recall the definition of $\vhi^s(q^1,\dots, q^i,\dots,q^n) = (q^1,\dots, q^i+s,\dots,q^n)$ in local coordinates. We observe that in the canonical coordinates on $T_qM$, we have that the vector field generated by $\vhi^s$ is $X(\vec q) = \pd{q^i}\big|_{\vec q}$
\begin{ex}
    Classical momentum is an Integral of Motion
\end{ex}
Consider $N$ particles in $\R^3$. Let $L=T-U$ where $T$ is kinetic energy and $U$ is potential energy. Assume that $U$ depends only on the differences $\vec{x}_a - \vec{x}_b$ for $a,b\in [N]$. In particular, $\pdof{U}{\dot{x}^i_\al} = 0$. Then for all $\vec{e}\in\R^3$, 
\[\vhi^s(\vec x_1,\dots,\vec x_N) = (\vec x_1+s\vec e,\dots,\vec x_N+s\vec e)\]
is a one-parameter group of symmetries of $L$.
The generator of $\vhi^s$ is $X=(\vec e,\dots,\vec e)$, and, as defined in Noether's theorem,
\begin{align*}
    I(\vec x_1,\dots,\vec x_N,\dot{\vec{x}}_1,\dots,\dot{\vec{x}}_N) &= \pdof{L}{\dot q}X\\
    &= \sum_{\al=1}^N\sum_{i=1}^3\pdof{L}{\dot x^i_\al} e^i \\
    &= \sum_{\al=1}^N \sum_{i=1}^3m_\al\dot{x}^i_\al e^i\\
    &=  \ang*{\sum_{\al = 1}^N m_\al\vec x_\al, \vec e}.
\end{align*}
Viewing the momentum of the system as the vector $\vec p = \sum_{\al =1}^N \vec{p}_\al$, we observe that the above quantity is $\vec p \cdot \vec e$. Therefore, the total momentum, $\vec p$ is constant (since $\ang{\vec p, \vec{e}}$ for any $\vec e\in\R^3$), since we have that $\frac{dI}{dt} = 0$ by Noether's theorem. Therefore, for any trajectory, the momentum of the system is conserved.
\begin{ex}
    Rotations in $\R^3$
\end{ex}
Suppose, as above, that the potential energy $U$ is purely a function of distance between positions: $|\vec x_1 - \vec x_2|$. Now for any $\vec w\in \R^3$, consider the one-parametric group of rotations around this axis with angular velocity $\|\vec w\|$. We recall (from Lie group theory) that this group is given by $\{\exp(tA_{\vec{w}})\in SO_3(\R)\,|\,t\in\R\}$. Therefore, under the identification, the generator of this flow of rotations is the vector field $Y(\vec r) = A_{\vec{w}}\vec r$.
\nl
We use that $\mathfrak{so}(3) \cong \R^3$ as Lie algebras with multiplication on $\R^3$ given by the cross product. Specifically, for $\vec w = (w^1,w^2,w^3)$, we have that the map
\[(w^1,w^2,w^3)\mapsto A_{\vec w}=\begin{pmatrix}
    0 & -w^3 & w^2 \\
    w^3 & 0 & w^1 \\
    -w^2 & -w^1 & 0
\end{pmatrix},\]
is a Lie algebra isomorphism, and $A_{\vec w}\vec r = \vec w \times \vec r$.
\nl
Using Noether, we have that
\begin{align*}
    I &= \pdof{L}{\dot x^i_a}Y^i_a\\
    &=\sum_i\ang{m\dot{\vec r}_a,\vec w\times\vec r_a}
\end{align*}
\begin{rmk}
    Note that the quantity $\pdof{L}{\dot q} X$ is well-defined.
\end{rmk}
\section{Hamiltonian Mechanics}
In Lagrangian mechanics, we define a smooth map $L:TM \to \R$. For $(q,v)\in TM (q\in M, v\in T_qM)$, assume that for all $q\in M$, the map
$q\mapsto L(q,v)$ is strongly convex, i.e. that 
\[d^2_vL = \sum_{\frac{\partial^2L}{\partial v^i\partial v^j}dv^idv^j}\]
is a positive-definite quadratic form. (In Calculus of Variations, this is called the strong Legendre condition).
\nl
Let $H(p,q)= \max_{v\in T_qM}(p(v)-L(q,v))$ where $p\in T_q^*M$. Then $H$ is a function on the cotangent bundle $TM$. Let $h(p,q,v) = p(v)-L(q,v)$ (such that $H(p,q) = \max_{v\in T_qM} h(p,q,v)$). We note that since $H$ is a function on the cotangent bundle, $T^*M$. Functions on the cotangent bundle are called \textbf{Hamiltonians} (anaogous to Lagrangians, which are functions on the tangent bundle).
\begin{ex}
    Hamiltonian for motion in 1 dimension.
\end{ex}
Let $L = \frac{m\dot x^2}{2} - U(x)$ for $x\in M=\R$. Then, we observe that $p\in T^*_xM$ is a one-dimensional vector space, meaning that we may identify $p(\cdot):T_xM\to \R$ as multiplication by a constant $p$. Therefore, $h(p,x,v) = pv-\frac{mv^2}{2}+U(x)$. Then $\pdof{h}{v}\big|_{v=v_0} = (p-mv)|_{v=v_0} = 0$ implies that $p = mv_0$ and $v_0 = \frac{p}{m}$. Therefore,
\[H(p,x) = h(p,x,v_0) = \frac{p^2}{m} - \frac{m\frac{p^2}{m^2}}{2}+U(x) = \frac{p^2}{2m}+U(x).\]
The Hamiltonian corresponding to the Lagrangian $L$ is the total energy of the system. The exact same conclusion holds for motion of particles in $\R^3$.
\nl
\begin{note}
    Cotangent Bundle
\end{note}
We define $T^*M = \sqcup_{q\in M}T_q^*M$. Let $(q^1,\dots, q^n)$ be a local coordinate system on $M$. For some $q\in M$, we have that $\left((dq^1)_q,\dots,d(q^n)_q\right)$ forms a basis on $T_q^*M$. Any $p\in T_q^*M$ can therefore be written $p = p_1(dq^1)_q + \cdots + p_n (dq^n)_q$ (where $(dq^i)_q\left(\pd{q^j}|_{q}\right) = \delta^i_j$). Therefore, the point $(p,q)\in T^*M$ can be described by the coordinates $(p_1,\dots,p_n,q^1,\dots,q^n)$. These coordinates are considered the canonical coordinates on $T^*M$. Under this coordinate system (and canoncical coordinates on $TM$), if $v = \sum_{j=1}^n v^j\pd{q^j}$ and $p = \sum_{i=1}^n p_idq^i$, then $p(v) = \sum_{i=1}^n p_iv^i$.
\nl
Now, taking $v$ in local coordinates as above, we have that $p(v) = \sum_i p_iv^i$. Then, $h(p,q,v) = \sum_i p_iv^i - L(q,v)$. Then, by strong convexity, the maximum of $v\mapsto h(p,q, v)$ is attained at some $v=v_0$. Therefore, we know that $\pdof{h}{v}\big|_{v=v_0} = 0$., so in particular, for every $i$,
\begin{align*}
    0&=\pdof{h}{v^i}\big|_{v=v_0}\\
    &=\pd{v^i}\left[\sum_i p_iv^i - L(q,v)\right]\big|_{v=v_0}\\
    &=\pd{v^i}\left[\sum_i p_iv^i\right]\big|_{v=v_0} - \pdof{L}{v^i}\big|_{v=v_0}\\
    &=p_i - \pdof{L}{v^i}\big|_{v=v_0}
\end{align*}
Thus, $p_i = \pdof{L}{v^i}\big|_{v=v_0}$.
\begin{thm}[Equivalence of E-L Equation and Hamiltonian System]
    The curve $q(t)$ in $M$ is a solution of E-L, i.e. 
    \[\frac{d}{dt}\pdof{L}{\dot q}(q(t),\dot q(t)) = \pdof{L}{q}(q(t),\dot q(t)),\]
    if and only if the curve $(q(t),p(t))\in T^*M$, where $p(t) = \pdof{L}{\dot q}(q(t),\dot q(t))\in T^*M$ is the solution to the following system:
    \[\left\{\begin{aligned}
        \dot q(t) &= \pdof{H}{p}(p(t),q(t))\\
        \dot p(t) &= -\pdof{H}{q}(p(t),q(t))
    \end{aligned}\right.\]
    where $H(p,q)$ is the Hamiltonian associated with the Lagrangian $L$.
\end{thm}
\end{document}
